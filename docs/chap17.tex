\chapter{Image Simulation}
\begin{rawhtml}
<!-- linkto simimgs.html -->
\end{rawhtml}

%
% reformatted and edited
% moved TEMPLATE here from Chapter 15, since its function is more
% correctly image simulation
% did *some* keyword checking, hard to tell if I got everything
% function-wise.
% rwp/osu 98Jul26

The VISTA image simulation commands are:
\begin{example}
  \item[PHOTONS\hfill]{add artificial stars and/or noise to an image}
  \item[TEMPLATE\hfill]{generate an image from a user-supplied profile}
  \item[DEVAUC\hfill]{generate an image with a deVaucouleurs profile}
\end{example}

\section{PHOTONS: Add Artificial Stars and/or Noise to an Image}
\begin{rawhtml}
<!-- linkto photons.html -->
\end{rawhtml}

\index{Photometry!Artificial Stars }
\index{Photometry!Artificial Images}
\begin{command}
  \item[Form: PHOTONS source [NSTARS=] [AT=r,c] [COUNTS=c1,c2] [MEAN=]\hfill]{}
  \item{[RN=] [GAIN=] [GAUSS] [FILE=] [FW=] [DAOPSF=]}
  \item{[PHOT] [NEW] [PSFLIB=] [STR=] [NDIV] [TRUNC]}
  \item[NSTARS=]{gives the number of stars to make.}
  \item[AT=R,C]{Put a single star at row R and column C
       This cannot be used with NSTARS.}
  \item[STR=file]{specifies a file with the positions and
       counts of the stars to be added}
  \item[COUNTS=a,b]{gives ranges in total counts for the
       stars. If only one number is given, range will be from 0 to that number}
  \item[MEAN=]{gives the mean level (counts) added to the image to mimic
       sky. If mean>=0, photon noise will be added properly given counts in
       each pixel.  Mean=-1 if unspecified. If mean<0 then no photon noise
       is added to image. }
  \item[RN=]{ gives the readout noise (electrons) 0 if unspecified. }
  \item[GAIN=]{gives the conversion between photons and counts.  This is in
       the units photons per count. Needed to do noise properly.}
  \item[GAUSS ]{tell program to make images from a sampled Gaussian, with
       FWHM specified by FW= keyword If not include, PSF profile will be
       read a a disk file specified by FILE= keyword}
  \item[FW=]{star FWHM in pixels for Gaussian, or half width for arbitrary
       PSF (in disk file)}
  \item[FILE=]{ specifies file with 250 numbers (arr(i),i=1,250)
       representing 1-D PSF: half-width (no. of pixels per 250 bins) must
       be specified with FW keyword.  If FILE is specified, GAUSS is
       ignored}
  \item[DAOPSF=file]{creates stars with PSF from a DAOPHOT .PSF file}
  \item[PSFLIB=file]{Specifies PSFLIB library file to use to look up PSF at
       various pixel centerings. Must be created with PSFLIB command and
       special image}
  \item[PHOT]{store the positions and total counts of added
       stars in a photometry file. }
  \item[NEW]{creates a new photometry file.  NEW implies PHOT}
  \item[NDIV]{allows user to change the pixel subdivisions
       used for the PSF integration over pixels}
  \item[TRUNC]{Makes PHOTONS integer truncate}
\end{command}

PHOTONS creates artificial stars and/or noise on an already existing
image. If no image exists, use CREATE to make one. It makes as many stars
as you want with whatever brightnesses you want. You can specify where to
put each star, or you can place them randomly on the image, or you can read
in positions from an external file. You can use a coarse Gaussian to
represent the PSF with whatever FWHM, or you can give a PSF to the program
to read from a preexisting disk file. You can use a DAOPHOT style PSF to
create images, or a PSFLIB file. You can store the information (position
and counts) about the created stars in a VISTA photometry file.

Without preparing an external file, you can either create one star at a
specified position or one or more stars at random positions. For a
specified position,use AT=r,c; if this keyword isn't used, positions will
be random on the frame, unless the STR= keyword is used. Specify the number
of stars desired using NSTARS=n.The number of counts in each star is
determined from the COUNTS=c1,c2 keyword. Each star will have a total
number of counts randomly distributed in the range c1 to c2. If you want a
star to have a particular number of counts c1, just set c2=c1+1.

To make stars from a list on disk, prepare a file with four columns:
ID,COL(X), ROW(Y), COUNTS.  Any format is OK, as long as there are spaces
between the numbers. Specify the filename with STR=filename and your stars
will be made.

The PSFs of the stars can be a Gaussian of whatever FWHM you desire, or it
can be integrated external disk file with a 1-D PSF, or you can use a PSF
created by the DAOPHOT PSF routine to create stars, or you can specify a
PSF library file to use. For a Gaussian PSF, use the GAUSS keyword along
with FW=FWHM to specify the FWHM. For an arbitrary symmetric PSF from a
disk file, use FILE=file to specify the file name. This file can contain up
to 1000 numbers with the profile of your desire tabulated. The program
converts this profile to a pixel scale using the FW= keyword to tell it how
many pixels corresponds to the NTABLE entries. The file should be a
formatted disk file written with (Fortran):
\begin{verbatim}
  WRITE(*)  NTABLE
  WRITE(*) (PSF_TABLE(I), I = 1, NTABLE)
\end{verbatim}
Note that the ``formatted'' write statement in Fortran makes a file that is
machine-dependent, and such files are generally not exportable to other
machine/operating-system architectures.

To use a DAOPHOT PSF, use the DAOPSF=file to specify the file name of the
DAOPHOT .PSF file.

For all the file options, the V\_DATADIR is the default directory.

Unless requested, noise will not be added to the image. To add noise, use
MEAN=mean and/or RN=rn. MEAN adds a constant value to all pixels in the
image, then computes noise for each pixel and adds it on. RN will add
readout noise to the image. Note that mean and rn are to be given in
counts. To get the correct noise, you also need to specify a gain (with
GAIN=gain) in units of photons per count. Recall, if MEAN is greater than
or equal to 0, noise will be added. Consequently if you want to add stars
to an image more than once, you will add photon noise each time if MEAN
is >= 0.  The default values are -1 for mean (no noise added), 0 for RN (no
noise added), and 1 for GAIN. The most useful technique is to add sets of
stars with MEAN=-1, and when all stars are finally placed, run PHOTONS with
MEAN=the desired sky level.

To store the information about the created stars (their positions and
brightnesses) in a VISTA photometry file, use the PHOT keyword. For a new
file, use the NEW keyword (NEW implies PHOT).


\section{TEMPLATE: Generate an Image from a User-Supplied Profile}
\begin{rawhtml}
<!-- linkto template.html -->
\end{rawhtml}
\index{Template!Generating model galaxy image}
\begin{command}
  \item[Form: TEMPLATE dest source {[PA=n]} {[PAM=n]} {[E=n]} {[FIT=n1,n2]}
       {[SUB]} {[GAUSS]} {[EXP]} {[HUB]} {[DEV]}\hfill]{}
  \item[dest]{(integer or \$ construct) is an image which already 
       exists to which the template will be written,}
  \item[source]{(integer or \$ construct) is the spectrum that
       contains the profile used to generate the template,}
  \item[PA=n]{(re)sets the position angle of radial profile
       extended object,}
  \item[PAM=n]{(re)sets the position angle of the major axis
       of the original image,}
  \item[FIT=n1,n2]{requests the template be generated by fitting
       the profile spectrum over pixels n1 through n2
       with the appropriate surface brightness law,}
  \item[SUB]{requests that the template be subtracted from
       image already in image buffer specified,}
  \item[GAUSS]{generates a Gaussian template (point source),}
  \item[EXP]{generates an exponential disk template (spiral),}
  \item[HUB]{generates a Hubble template (elliptical),}
  \item[DEV]{generates a deVaucouleurs template (elliptical).}
\end{command}

TEMPLATE generates either a Gaussian (GAUSS), exponential disk (EXP),
Hubble (HUB), or deVaucouleurs (DEV) template from an input profile spectrum
originally calculated using the CUT or PROFILE command.  The Eccentricity,
Position Angle of the brightness profile stored in the spectrum, and the
Position Angle of the Major axis are all stored in a common block and
generally need not be reloaded using the appropriate keywords.  These
keywords permit the user to "fine tune" the fit in the event that the
loaded parameters are not ideal.  The actual nonlinear least squares fit to
the appropriate brightness law is carried out for spectrum points n1
through n2 using the FIT= keyword.  This permits, for example, fitting an
exponential disk to the outer portions of the brightness profile while
leaving the central (perhaps bulge) component alone.  At this time only one
surface brightness law at a time can be fit.

N.B.The user must have an image with the same characteristics (size, etc.)
as the template already in the destination buffer.  This image will of
course be overwritten by the generated template.  The SUB keyword permits
the subtraction of the generated template from the image already in the
destination buffer.

\noindent{Example:}
\begin{example}
  \item[TEMPLATE 2 1 FIT=10,19 EXP\hfill]{Generates an exponential disk
       template from the profile in spectrum 1 (fit to pixels 10 through
       19) and writes the image to image buffer 2 (which already exists).}

  \item[TEMPLATE 2 1 FIT=10,19 DEV SUB \hfill]{same as above, but fits
       deVaucouleurs profile and subtracts fit from image already in 2}
\end{example}

See Also: CUT, DEVAUC


\section{DEVAUC: Generate an Image with a DeVaucouleurs Profile}
\begin{rawhtml}
<!-- linkto devauc.html -->
\end{rawhtml}
\begin{command}
  \item[Form: DEVAUC source [REFF=r] [SEFF=s] [X0=x0] [Y0=y0] 
       [PIX=pix]\hfill]{}
  \item[source]{gives the image in which to create the simulation}
  \item[REFF=reff]{specifies the effective radius (arcmin)}
  \item[SEFF=seff]{specifies the surface brightness at reff (counts/pixel)}
  \item[X0=]{specifies the column center}
  \item[Y0=]{specifies the row center}
  \item[PIX=pix]{specifies the pixel scale ("/pixel)}
\end{command}

DEVAUC will generate an object with a deVaucouleurs profile in buffer
source. The buffer must have previously been created. The parameters of the
profile can be specified using the keywords, or else the program will
prompt the user for them.

See Also:  TEMPLATE
