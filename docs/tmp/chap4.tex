\chapter{VISTA Command Procedure Scripts}

%
% minor editing and reformatting
% used verbatim environment extensively for example scripts
% rwp/osu 98Jul27
%

\section{Introduction to Command Procedure Scripts}
\begin{rawhtml}
<!-- linkto procedure.html -->
<!-- linkto procedures.html -->
\end{rawhtml}

\index{Procedure!Defined}
\index{Procedure!Introduction to}
\index{Procedure!Startup procedure}

VISTA can store several commands in a list and execute them as a program.
A list of such commands is termed a PROCEDURE.  The list is stored in a
special buffer, called the 'procedure buffer'.

Almost any VISTA command that has proper syntax can be used in a procedure.
The basic commands for creating, storing, and modifying procedures are
these:
\begin{itemize}
  \item[PEDIT\hfill]{edits the current procedure buffer.}
  \item[WP\hfill]{stores the procedure buffer on the disk.}
  \item[RP\hfill]{reads the procedure from disk.}
  \item[SHOW\hfill]{displays the procedure buffer}
  \item[GO\hfill]{begins execution of the procedure.}
\end{itemize}

There are several 'control commands' that effect the operation of a
procedure.
\begin{itemize}
  \item[VERIFY\hfill]{executes a procedure line by line, to aid in debugging.}
  \item[PAUSE\hfill]{pauses during execution of a procedure.}
  \item[CALL\hfill]{runs a procedure as a subroutine.}
  \item[RETURN\hfill]{returns from a procedure used as a subroutine.}
  \item[DO, END\_DO\hfill]{define a loop in a program for execution a given
       number of times,}
  \item[GOTO\hfill]{jumps to another place in the procedure.}
  \item[:\hfill]{defines a place to jump to in the procedure.}
  \item[IF, END\_IF \hfill]{define a block of commands that are executed
       only under certain conditions.}
  \item[ELSE, ELSE\_IF\hfill]{control branching for branching that has many
       options.}
\end{itemize}

This list serves not only as an introduction to those not familiar with
procedures, but illustrates the flexibility that procedures give to VISTA
programming.  A defined procedure eliminates the boredom of typing
repetitive commands over and over, but it does much more than that: it
greatly expands the functions of VISTA so that new applications do not
always require new subroutines.  A familiarity with procedures will make
your data reductions more efficient.

\noindent{IMPORTANT:}\newline
Do NOT put any of the flow-control statements (DO, IF, END\_DO, etc.) with
other commands on a line.  They must appear separately.  Thus:
\begin{verbatim}
   DO I=1,10}
   DO J=1,10}
      (some stuff)}
   END\_DO; END\_DO
\end{verbatim}
or similar constructions, will NOT work properly!

As the VISTA program begins, it executes the procedure stored in the file
defined by the environment variable V\_STARTUP.  For example, if you had
defined V\_STARTUP through
\begin{verbatim}
   setenv V_STARTUP ./myprocs/myproc.pro
\end{verbatim}
\textit{before} running VISTA, then myprocs/myproc.pro will be executed as the
program begins.  Typically, the startup procedure will contain definitions
of aliases, the setting of symbol values, or the reading into buffers of
repeatedly-used images.  This procedure is not saved in the procedure
buffer as it is executed.

\section{PEDIT: Edit the Procedure Buffer}
\begin{rawhtml}
<!-- linkto pedit.html -->
\end{rawhtml}

\begin{itemize}
  \item[\textbf{Form: } PEDIT\hfill]{}
\end{itemize}

PEDIT loads the procedure buffer into a temporary file in your current
directory, then runs a process which allows you to edit it with the default
editor (usually vi).  If you exit the editor and save changes, the modified
procedure is loaded back into the procedure buffer, but not executed.  If
quit the editor without saving changes, the procedure buffer is left
unchanged.

On a UNIX system, you can use another editor by defining the environment
variable VISUAL. For example, to use the vi editor, jus execute the
following BEFORE running VISTA:
\begin{verbatim}
   setenv VISUAL vi
\end{verbatim}
A maximum number of 2500 lines is allowed in a procedure.  If you leave the
editor with EXIT and your procedure contains more than 2500 lines, it is
truncated and the last line is set to 'END'.  Also, each line is allowed a
maximum length of 80 characters. You can have longer lines by using
\htmladdnormallink{linecontinue.html}{line continuation} as described
earlier.

You may also define procedures (with a maximum length of 2500 lines) with
the editor before running VISTA, and read them into the procedure buffer
with RP or execute them with CALL.

\section{STOP: Stop Procedure Execution}
\begin{rawhtml}
<!-- linkto stop.html -->
\end{rawhtml}

\index{Procedure!Ending execution}
\begin{itemize}
  \item[\textbf{Form: } STOP {['A message']}\hfill]{}
\end{itemize}
The STOP command causes a procedure to stop executing.  If the message is
supplied, it will be typed on the terminal, otherwise a default message is
typed.  A stack unwind is also produced if the stop occurs in a CALLed
procedure.

\section{END: End Procedure Execution}
\begin{rawhtml}
<!-- linkto end.html -->
\end{rawhtml}

\index{Procedure!Ending definition}
\index{Procedure!Ending execution}
\begin{itemize} 
  \item[\textbf{Form: } END\hfill]{}
\end{itemize}

All procedures must end with END or RETURN.

When this command is entered during a procedure definition it tells VISTA
to leave the procedure-definition mode and to return to the
command-execution mode.  The command is also saved in the procedure buffer
and signals the end of the current procedure when it is executed.  If the
procedure is executed as a subroutine, the END command, like RETURN (q.v.),
tells VISTA to return to the calling procedure.


\section{SHOW: Show the Current Procedure Buffer}
\begin{rawhtml}
<!-- linkto show.html -->
\end{rawhtml}

\index{Procedure!List}
\index{Procedure!Print}
\begin{itemize}
  \item[\textbf{Form: } SHOW {[output redirection]}\hfill]{}
\end{itemize}

This command lists out the lines or commands held in the procedure buffer.
The SHOW command takes no arguments.  The output may be redirected, as in
\begin{itemize}
  \item{SHOW $>$LP:}
\end{itemize}

\section{WP: Write a Procedure to Disk}
\begin{rawhtml}
<!-- linkto wp.html -->
\end{rawhtml}

\index{Procedure!Write to Disk}
\index{Disk!Write procedure to}
\begin{itemize}
  \item[\textbf{Form: } WP filename\hfill]{}
  \item[filename]{is the name of the file that will store the current procedure.}
\end{itemize}

Unless otherwise specified in 'filename', WP will write to the default
directory for procedures, which may be displayed by the command PRINT DIR,
and uses the default extension.  As an example, suppose the default
procedure directory is /vista/procedure and the default extension is .pro

\begin{itemize}
  \item[WP medfly\hfill]{writes the current procedure to the file
       vista/procedure/medfly.pro}
  \item[WP /demo/medfly\hfill]{writes to /demo/medfly.pro}
  \item[WP medfly.xyz\hfill]{writes to vista/procedure/medfly.xyz}
\end{itemize}

\section{RP: Read a Procedure from Disk}
\begin{rawhtml}
<!-- linkto rp.html -->
\end{rawhtml}

\index{Procedure!Read from disk}
\index{Disk!Read procedure from}
\begin{itemize}
  \item[\textbf{Form: } RP filename\hfill]{}
  \item[filename]{is the name of the file that holds the desired procedure.}
\end{itemize}

RP will read a maximum of 2500 lines, with a maximum of 80 characters per
line, from the designated filename into the procedure buffer.  Unless
otherwise specified in 'filename', RP will read from the default directory
for procedures, which may be displayed by the command PRINT DIR, and uses
the default extension.  As an example, suppose the default procedure is
/vista/procedure and the default extension is .pro

\noindent{ Examples:}
\begin{itemize}
  \item[RP medfly\hfill]{reads the contents of /vista/procedure/medfly.pro
       into the procedure buffer.}
  \item[RP /demo/medfly\hfill]{reads from /demo/medfly.pro}
  \item[RP medfly.xyz\hfill]{reads from /vista/procedure/medfly.xyz}
\end{itemize}

\section{GO: Start Procedure Execution}
\begin{rawhtml}
<!-- linkto go.html -->
\end{rawhtml}

\index{Procedure!Execute}
\index{Procedure!Run}
\begin{itemize}
  \item[\textbf{Form: } GO {[parameter1]} {[parameter2]} ...]{}
  \item[parameter1,2,...]{are parameters passed to the procedure.}
\end{itemize}
GO tells VISTA to start executing the procedure held in its procedure
buffer.  You may also supply parameters to the procedure on the command
line.  The parameters must be evaluated using the PARAMETER command in the
procedure.

\noindent{Examples:}
\begin{itemize}
  \item[GO 10\hfill]{executes the procedure in the buffer,
       passing the numeric parameter 10 to the procedure.}
  \item[GO mydir/image\hfill]{execute the procedure, passing the
        string parameter to the procedure.}
\end{itemize}

Earlier versions of VISTA allowed you to specify a number of times to
execute a procedure, as in 'GO 3', which executed the current procedure
three times.  This was installed in the earliest versions of VISTA, before
there were DO loops.  This feature has been deleted from the version 3.
The old version 2 form 'GO filename' has been eliminated, since this can
now be done with the CALL command.

\section{CALL: Call a Procedure as a Subroutine}
\begin{rawhtml}
<!-- linkto call.html -->
\end{rawhtml}

\index{Procedure!Jump to subroutine}
\index{Procedure!Subroutine}
\index{Subroutine!Calling}
\begin{itemize}
  \item[\textbf{Form: } CALL procedure\_filename {[parameter1]} {[parameter2]} ...\hfill]{}
  \item[procedure\_filename]{is the name of a file holding a procedure.}
  \item[parameter1,2,...]{are optional parameters passed to the called procedure.}
\end{itemize}

CALL tells VISTA to save the contents of its current procedure buffer, read
in the desired procedure file, and begin execution at its first line.  The
CALL command can be executed directly in the immediate input mode, or be
used inside procedures to call other procedures.  In both cases, at the
completion of the called procedure, VISTA will return properly to either
the input mode or calling procedure.  VISTA will support up to 10 levels or
subroutine calls.  If an error occurs while a called procedure is
executing, VISTA will unwind and display the complete subroutine stack.

A procedure is allowed a maximum length of 2500 lines, and each line is
allowed a maximum length of 80 characters.  If your procedures or lines
exceed this maximum, they will be truncated.

Parameters can optionally be passed to the procedure.  These can be numeric
or string parameters.  The parameters are evaluated by the procedure using
the PARAMETER command.

Examples:  Assume the default procedure directory is /vista/procedure
and the default extension is .pro.
\begin{itemize}
  \item[CALL medfly\hfill]{executes /vista/procedure/medfly.pro}
  \item[CALL /mydir/medfly\hfill]{executes /mydir/medfly.pro}
  \item[CALL medfly.txt\hfill]{executes /vista/procedure/medfly.txt}
  \item[CALL medfly kill\hfill]{executes /vista/procedure/medfly.pro
passing it the parameter KILL}
\end{itemize}

\section{RETURN: Return from a Procedure Subroutine}
\begin{rawhtml}
<!-- linkto return.html -->
\end{rawhtml}

\index{Procedure!Return from subroutine}
\begin{itemize}
  \item[\textbf{Form: } RETURN\hfill]{}
\end{itemize}

This command tells VISTA that the execution of the current sub-procedure is
complete and to return to any calling procedure or to immediate input mode
as is appropriate.  This command is intended to allow a return from the
procedure as a result of an IF test. In cases where no condition testing is
needed, the final END command in the procedure buffer will tell VISTA that
the procedure has completed.

\noindent{Examples:}
\begin{enumerate}
  \item{Here is an example of a conditional exit from a procedure.
   This procedure may be run with CALL or GO.
  \begin{verbatim}
     LOOP:
        ASK 'Enter non-zero to process another image >> ' TEST
        IF TEST=0
           RETURN
        END_IF
         :
         :
     GOTO LOOP:
  \end{verbatim}
  }
  \item{END may be used as return ONLY if it is the last command in 
  the file. Thus a procedure like:
  \begin{verbatim}
     (commands)
     END
  \end{verbatim}
  may be executed as a subroutine, but one that looks like:
  \begin{verbatim}
     IF TEST=0
        END
     END_IF
  \end{verbatim}
  will not work.  Use RETURN instead.
}
\end{enumerate}

\section{PARAMETER: Pass Arguments to a Procedure}
\begin{rawhtml}
<!-- linkto parameter.html -->
\end{rawhtml}

\index{Procedure!Parameter Passing}
\begin{itemize}
  \item[\textbf{Form: } PARAMETER {[varname]} {[varname]} {[STRING=string\_var]} ...\hfill]{}
\end{itemize}
PARAMETER evaluate parameters passed to a procedure.  An example will show
best how PARAMETER works.  Suppose the following command was typed:
\begin{itemize}
  \item{CALL TEST 2 X IMAGEFILE}
\end{itemize}
The parameter list portion of the command line is saved, i.e.  '2 X
IMAGEFILE'.  The procedure TEST.PRO is then called.  TEST has in it the
command
\begin{itemize}
  \item{PARAMETER BUFNUM FACTOR STRING=FILENAME}
\end{itemize}
BUFNUM and FACTOR are taken to be variable names and are associated with
the first two parameters from the parameter list.  In effect it does the
command SET BUFNUM=2 FACTOR=X.  Note that '2' and 'X' could have been any
arithmetic expression.

The STRING= keyword means that the third parameter is used as a literal
string and the string variable name FILENAME is given the string value
IMAGENAME, in effect doing the command:
\begin{itemize}
  \item{STRING FILENAME IMAGENAME}
\end{itemize}

If there are fewer parameters in the CALL than required in the PARAMETER
command, the missing parameters are taken to be 0 or blank, as appropriate.

Note that there is only one area in VISTA for saving the list of passed
parameters, so the PARAMETER command should be executed before calling
another procedure.

\section{VERIFY: Verify (Trace) Procedure Execution}
\begin{rawhtml}
<!-- linkto verify.html -->
\end{rawhtml}

\index{Procedure!Line by line execution}
\index{Procedure!Debugging}
\begin{itemize}
  \item[\textbf{Form: } VERIFY Y or VERIFY N\hfill]{}
\end{itemize}
VERIFY causes each line of the procedure to be shown on the terminal just
prior to its execution, allowing you to watch the procedure work line by
line.  The keyword 'Y' turns the display on and the keyword 'N' turns the
display off.

\section{DEF: Define a Procedure}
\begin{rawhtml}
<!-- linkto def.html -->
\end{rawhtml}
\index{Procedure!Defining}
\begin{itemize} 
  \item[\textbf{Form: } DEF [line\_number]\hfill]{}
  \item[line\_number]{is the [optional] line number of the
beginning of the new definition.}
\end{itemize}

Archaic command, use PEDIT instead.

DEF tells VISTA to begin a procedure definition. It is mostly historical,
as procedures can now be defined using a real editor with the PEDIT
command. The command DEF will prompt you with a series of line numbers,
beginning with the number specified in the command line.  If no number was
given, the first command will be on line 1.  You are to specify a command
for each line.  Type the command you want on that line, then hit RETURN.
VISTA will check the command for proper syntax, and store it in the
procedure buffer. Then VISTA will prompt you with the number of the next
line.  Continue typing in commands until the entire procedure has been
entered.  To tell VISTA that you have entered the entire procedure, type
END or SAME (q.v.).  A maximum length of 2500 lines is allowed in a
procedure, and each line is allowed a maximum length of 80 characters.

\noindent{Examples:}
\begin{itemize}
  \item[DEF\hfill]{begins a procedure definition on line 1.}
  \item[DEF 10\hfill]{begins a procedure definition on line 10.  
       Commands on lines 1 through 9, if there are any, are preserved.}
\end{itemize}

\section{SAME: End Procedure Insertion and Keep Trailing Lines}
\begin{rawhtml}
<!-- linkto same.html -->
\end{rawhtml}

\index{Procedure!Ending definition}
\begin{itemize} 
  \item[\textbf{Form: } SAME\hfill]{}
\end{itemize}

Archaic command: see PEDIT

When this command is entered during a procedure definition it tells VISTA
to leave the procedure-definition mode and to return to the
command-execution mode.  Unlike the END command, however, the SAME command
tells VISTA to keep any lines that may have been defined after the
insertion point, that is, the commands in the buffer following the
insertion point are to be left the same as they were before.  The SAME
command makes sense only when you are modifying previously defined
procedures, and is not saved in the procedure buffer. It is also largely
historical, as it is not needed if you use the PEDIT command to modify
procedures.

