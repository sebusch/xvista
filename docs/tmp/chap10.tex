\chapter{Image Arithmetic}

\section{Introduction to Image Arithmetic}
\begin{rawhtml}
<!-- linkto math.html -->
\end{rawhtml}

%
% all entries checked for keywords & function against current sources
% added missing entries for various functions (e.g., ONEOVER)
% 
% rwp/osu 1998 Aug 14
%

An ``image'' may be either a 2-D image array or a 1-D spectrum vector.  The
commands for one- or two-image arithmetic are:

\begin{itemize}
  \item[ADD\hfill]{add two images or an image and a constant.}
  \item[SUBTRACT\hfill]{subtract two images or subtract a constant and a 
       image.}
  \item[MULTIPLY\hfill]{multiply two images or an image and a constant}
  \item[DIVIDE\hfill]{divide two images or divide an image by a constant.}
  \item[SQRT\hfill]{square root}
  \item[LOG\hfill]{base-10 logarithm}
  \item[LN\hfill]{natural logarithm}
  \item[EXP\hfill]{exponentiate image (e\^image):  LOG INV does 10\^image}
  \item[TAN\hfill]{tangent}
  \item[ARCTAN\hfill]{inverse of TAN}
  \item[SIN\hfill]{sine}
  \item[COS\hfill]{cosine}
  \item[NINT\hfill]{nearest integer}
  \item[ONEOVER\hfill]{invert (1/x)}
\end{itemize}

\section{ADD, SUBTRACT, MULTIPLY, and DIVIDE}
\begin{rawhtml}
<!-- linkto add.html -->
<!-- linkto subtract.html -->
<!-- linkto multiply.html -->
<!-- linkto divide.html -->
\end{rawhtml}
\index{Image!Add two images}
\index{Image!Add image and constant}
\index{Image!Subtract two images}
\index{Image!Subtract constant from image}
\index{Image!Multiply two images}
\index{Image!Multiply image by constant}
\index{Image!Divide two images}
\index{Image!Divide image by constant}
\begin{itemize}
  \item[\textbf{Form:}ADD        dest {[other]} {[CONST=c]} {[BOX=B]} {[DR=dr]} {[DC=dc]}\hfill]{}
  \item[SUBTRACT   dest {[other]} {[CONST=c]} {[BOX=B]} {[DR=dr]} {[DC=dc]}\hfill]{}
  \item[MULTIPLY   dest {[other]} {[CONST=c]} {[BOX=B]} {[DR=dr]} {[DC=dc]}\hfill]{}
  \item[DIVIDE     dest {[other]} {[CONST=c]} {[BOX=B]} {[DR=dr]} {[DC=dc]} {[FLAT]}\hfill]{}
  \item[dest]{is the buffer where the result will be stored}
  \item[other]{is the other buffer in the calculation, if any.}
  \item[CONST=c]{performs arithmetic between a buffer and a constant}
  \item[BOX=B]{limits the arithmetic to those pixels
in the source which are in box B}
  \item[DR=dr]{Gives a row offset between the 'dest' and 'other' buffers.}
  \item[DC=dc]{Gives a column offset between the 'dest' and 'other' buffers.}
  \item[FLAT]{rescales the division by the mean of 'other', preserving the 
mean of the source.}
\end{itemize}
These commands perform all arithmetic between images and spectra.  There
are three ways to use these commands.

\noindent{\textbf{Operations between two images or two spectra:}}

In this form, the command word is followed by the numbers of the buffers
holding the images.  Examples are
\begin{itemize}
  \item[ADD 1 2\hfill]{Add image in buffer 2 to image in buffer 1.}
  \item[SUBTRACT 3 4\hfill]{Subtract image in buffer 4 from image in buffer 3.}
\end{itemize}
Note that the first buffer listed is the one modified by the operation.

The FLAT word is used in the DIVIDE command to re-scale the division by the
mean of the second image, thus preserving the mean of the first buffer.
This is typically used when dividing raw images by a flat field.
\textbf{NOTE:} The mean of the second image must have previously been
calculated by the MN command!

\noindent{\textbf{Operations between images and constants:}}

In this form, there is only one buffer specification on the command line.
The word CONST= (or C=) is used to specify the constant to be used.
\begin{itemize}
  \item[ADD 2 CONST=9.5\hfill]{Adds 9.5 to every pixel in buffer 2}
  \item[DIVIDE 3 CONST=10/3.1415\hfill]{  Divides buffer 3 by 10/3.14159}
\end{itemize}

\noindent{\textbf{Combined image and constant operations:}}

The first two forms of these commands may be combined, thus allowing you to
simultaneously operate on two images and a constant.
\begin{itemize}
  \item[ADD 2 3 CONST=5\hfill]{Adds buffers 3 to 2, then adds 5.0
       to the result.}
  \item[MULTIPLY 1 \$J CONST=0.01\hfill]{ Multiplies buffer 1 by buffer J
       (J a variable), then multiplies the result by 0.01.}
\end{itemize}

\noindent{\textbf{Remarks:}}

In two-image arithmetic, only those pixels common to the two images are
changed by the operation.  The size of the destination image is not changed
by the operation.  The DR= and DC= keywords only apply to two-image
arithmetic and specify row and column offsets between the two images.  The
sense of the shift is shown in the following example: dest(C+DC,R+DR) =
dest(C+DC,R+DR) + other(C,R).

BOX limits the operation to those pixels in the destination image which are
in the specified box.

\section{SQRT: Square Root of an Image}
\begin{rawhtml}
<!-- linkto sqrt.html -->
\end{rawhtml}
\index{Image!Square root}
\begin{itemize}
  \item[\textbf{Form:}SQRT source {[SGN]} {[SIGN]} {[NOABS]}\hfill]{}
\end{itemize}

SQRT replaces each pixel in 'source' by the square root of the pixel value.
If there are negative pixels, the square root of the absolute value is
taken, unless the NOABS keyword is given, in which case negative pixels are
set to zero, or unless the words SGN or SIGN are included.  If either if
these words are included on the command line, the square root of the
absolute value is computed and given a negative sign (example: -4 turns
into -2).

\section{LOG: Base-10 Logarithm of an Image}
\begin{rawhtml}
<!-- linkto log.html -->
\end{rawhtml}
\index{Image!Logarithm}
\begin{itemize}
  \item[\textbf{Form:} LOG source {[INV]}\hfill]{}
\end{itemize}

LOG replaces the pixel values in the image 'source' by their base-10
logarithm values. Any pixels whose original values were less than or equal
to 0 are replaced by -30 in the resulting image.  If the INV keyword is
specified, the inverse log is taken

\noindent{Example:}
\begin{itemize}
  \item[LOG 3\hfill]{take the log of each pixel in buffer 3.}
\end{itemize}

\section{LN: Natural Logarithm of an Image}
\begin{rawhtml}
<!-- linkto ln.html -->
\end{rawhtml}
\index{Image!Logarithm}
\begin{itemize}
  \item[\textbf{Form :} LN source {[INV]}\hfill]{}
\end{itemize}

LN replaces the pixel values in the image 'source' by their base-e
logarithm values. Any pixels whose original values were less than or equal
to 0 are replaced by -30 in the resulting image.  If the INV keyword is
specified, the inverse log is taken

\noindent{Example:}
\begin{itemize}
  \item[LN 3\hfill]{take the natural log of each pixel in buffer 3.}
\end{itemize}

\section{EXP: Exponentiate an Image}
\begin{rawhtml}
<!-- linkto exp.html -->
\end{rawhtml}
\index{Image!Exponentiation}
\begin{itemize}
  \item[\textbf{Form:}EXP source\hfill]{}
\end{itemize}

EXP replaces every pixel value n in 'source' by $e^n$.  Values larger than
+30 or smaller than -30 are set to 1.0E+30 or 1.0E-30 respectively.

\section{TAN: Tangent of an Image}
\begin{rawhtml}
<!-- linkto tan.html -->
\end{rawhtml}
\index{Image!Tangent}
\begin{itemize}
  \item[\textbf{Form:}  TAN source\hfill]{}
\end{itemize}

TAN replaces every pixel n in 'source' by tan(n).  The pixel values are
assumed to be in degrees.  All pixel values are mapped to the range -90 to
+90 degrees.

\section{ARCTAN: Arctangent of an Image}
\begin{rawhtml}
<!-- linkto arctan.html -->
\end{rawhtml}
\index{Image!Arctangent}
\begin{itemize}
  \item[\textbf{Form:} ARCTAN source {[0TO180]}\hfill]{}
\end{itemize}

ARCTAN replaces every pixel in source by the arctangent of the original
value.  The result of ARCTAN is in degrees from -90 to +90 by default.  If
you wish the arctangent values to lie between 0 and 180 degrees, use the
keyword 0TO180.

\section{COS: Cosine of an Image}
\begin{rawhtml}
<!-- linkto cos.html -->
\end{rawhtml}
\begin{itemize}
  \item[\textbf{Form:} COS source\hfill]{}
\end{itemize}

COS replaces every pixel in source by the cosine of the original value. The
pixel values are assumed to be in degrees.

\section{SIN: Sine of an Image}
\begin{rawhtml}
<!-- linkto sin.html -->
\end{rawhtml}
\begin{itemize}
  \item[\textbf{Form:} SIN source\hfill]{}
\end{itemize}

SIN replaces every pixel in source by the sine of the original value. The
pixel values are assumed to be in degrees.

\section{NINT: Nearest Integer of an Image}
\begin{rawhtml}
<!-- linkto nint.html -->
\end{rawhtml}
\begin{itemize}
  \item[\textbf{Form:} NINT source\hfill]{}
\end{itemize}

NINT replaces every pixel in source with the nearest integer value.

\section{ONEOVER: Invert an Image}
\begin{rawhtml}
<!-- linkto oneover.html -->
\end{rawhtml}
\begin{itemize}
  \item[\textbf{Form:} ONEOVER source\hfill]{}
\end{itemize}

ONEOVER replaces every pixel in source by its inverse (1/x).  If the pixel
value is 0, it sets the value to 0 in the inverse image to flag it.

