%
%  VISTA Cookbook;  VISTA and the FITS Standard (Appendix C
%
%  Began:  1988 August 9
%
%  Last Revision:  1988 August 31
%
%  Note:  Throughout this chapter, subsections are all labelled for
%         cross-referencing.  The base reference key is:
%
%                        sec:fits
%

%\documentstyle {manual}
%\input sys$user:[rick.thesis]thesismacros.pogge
%
%\newenvironment{command}{\begin{center}
%\begin{list}{\tt GO:}{\setlength{\rightmargin}%
%{\leftmargin}}\tt\singlespace }{\end{list}\end{center}}
%
%\def \comm#1{{\tt #1\/}}
%\def \hitkey#1 {$\langle${\tt #1}$\rangle$\/}
%
%\begin{document}

\chapter{VISTA and the FITS Standard}

This appendix is meant to discuss, briefly, the basic principles of the FITS
standard, and how VISTA interacts with it.  The main reason for this
discussion is to clarify areas where VISTA has adopted some of the optional
FITS header cards to somewhat non-standard applications, and troubles that can
arise reading data from other sites.   This primarily for experts, but since
VISTA has a \comm{FITS} command that allows the user to change FITS cards,
it is possible (and demonstrated) that the user can crash VISTA right to the
floor by tweaking a wrong card (``not {\it that} button!'').

To get straight to one important point.  In the current release of VISTA in
use since 1986, images written to magnetic tape by VISTA are FITS standard,
and have been read successfully by most other image processing programs in use
among the astronomical community (IRAF, MIDAS, and AIPS).  In addition, VISTA
has been able to read FITS tapes brought from other institutions (\eg\ ESO),
but it cannot at present read so-called ``blocked'' tapes written at 6250bpi
density.

As will be discussed below, VISTA has adopted a few of the standard FITS
header cards for use {\it internally} in a manner inconsistent with the
accepted standard.  This is perfectly harmless for most applications.  Where
VISTA collides with the standard is in operation which attempt to work in
``physical'' (or ``world'') coordinates --- such as arcseconds, wavelength,
etc. --- explicitly.  For most applications, like 1-D spectra, the difference
vanishes.  For more sophisticated applications, like calibrated 2-D spectra
(this does not include echelle spectra, but rather long-slit spectra), there
can be points of confusion.  It is hoped that this discussion will serve to
clarify the issues.

\section{The FITS Standard}

FITS is an acronym for {\it Flexible Image Transport System}.  It is an
internationally agreed upon structuring standard governing the format used to
write astronomical data to magnetic tape.  The basic intent is to make any
data written anywhere in the world completely portable to anywhere else,
without having to have a separate program to read the data.  One standard
format, one standard set of conventions, one program (in principle) capable of
reading everything.

For the most part, this has been accomplished.  Some uses of the FITS format
for storing multi-dimensional radio data (or Fabry-Perot Interferometer
images) are more than some sites can handle, but when it comes to 2-D images,
it works quite well.  I won't go into great detail in this appendix, but
rather refer you to the principal reference: {\it FITS: A Flexible Image
Transport System}, Wells, Greisen, and Harten, 1981, {\it Astronomy and
Astrophysics Supplement Series}, {\bf 44}, 363.

The basic structure of a FITS file, as it pertains to 2-D images, is as
follows:  Each file has a {\it Header} containing all of the information on
the structure of the image data on the tape, and any auxilliary information
describing the contents to the user (\eg\ object name, exposure time, date of
observation, etc.).  Following the Header are the data records, which contain
the image data itself.

The way most programs, including VISTA, work is to first read the header. This
will tell the program how to sort out the data records and essentially
reconstruct the image from an (effectively) unstructured stream of numbers.
The means that data storage is (in principle) as compact as possible, and
intelligible to any program that can interpret FITS headers.  This is the
essense of ``portability.''

The organization of the header is relevant to the following discussion. The
header is divided into ``cards'' each of which contains some piece of
information.  The first 8 characters of a card contain the name of the card.
The header cards most essential for image untangling have been standardized.
In brief, 6 cards are absolutely required for a 2-D image.  These are, IN
ORDER:

\begin{description}
      \item[SIMPLE] A logical variable, always ``T'' (True) if the
		    image is a ``simple'' FITS image.  All VISTA images are
		    ``simple'' in this regard.

      \item[BITPIX] An integer, it gives the number of bits per pixel
		    of the image data.  VISTA, and most other programs,
		    support only BITPIX=16 and BITPIX=32.

      \item[NAXIS]  An integer, equal to the number of image axes.  A 2-D
		    image has\hfil\break ``NAXIS=2''.  A 1-D image
		    (``spectrum'') has ``NAXIS=1''.

      \item[NAXIS1] The number of pixels along the `first' image axis.  For 2-D
		    images, this is the number of columns, or for spectra, the
		    number of pixels (sometimes called ``channels'').

      \item[NAXIS2] The number of pixels along the `second' image axis.  For
		    2-D images, this is the number of rows, and for spectra,
		    NAXIS2 is 1.

      \item[END]    In between the last {\it NAXISn} card and the end of the
		    FITS header, you can have as many cards with as much
		    info as you like.  However, the last card must be ``END''.
\end{description}

Beyond this basic set of FITS cards, all of the other cards are considered
optional.  FITS cards can take real, integer, or character variables, single
or double precision.  The FITS standard also defines some basic `optional'
cards that most people use.  Thus, while intended as `optional', they have by
common use become an integral part of the standard.  Again, interested readers
are referred to the aforementioned paper for details.  VISTA has adopted some
of these for its own purposes, and these are discussed in the next section.

\section{FITS Cards Important to VISTA}

VISTA uses some of the FITS header cards info in non-standard ways that can
potentially cause trouble if care is not taken.  These are listed in the table
below, with the VISTA usage and standard usage given.

\begin{center}
  \begin{tabular}{|l|p{2.0in}|p{2.0in}|}
    \hline
    \multicolumn{3}{|c|}{Special FITS Cards}\\ \hline
    \multicolumn{1}{|c|}{FITS Card}&
    \multicolumn{1}{c|}{VISTA Usage}&
    \multicolumn{1}{c|}{Standard Usage}\\ \hline
      CRVAL1&Starting Column of an Image&Value in physical coordinates
	    along the columns axis of the reference pixel in columns\\
      CRVAL2&Starting Row of an Image&Value in physical coordinates along the
	    rows axis of the reference pixel in rows\\
      CDELT1&On-Chip Binning Factor in Columns&pixel scale in physical
	    coordinates along columns axis\\
      CDELT2&On-Chip Binning Factor in Rows&pixel scale in physical coordinates
	    along rows axis\\
      CRPIX1&Array index of starting column&Array index of the refernce pixel
	    in columns\\
      CRPIX2&Array index of starting row&Array index of the refernce pixel
	    in rows\\
      CTYPE1&If a spectrum, indicates wavelength scale type&Units of physical
	    coordinates along the columns axis (\eg\ arcsec)\\
      CTYPE2&Unused&Units of physical coordinates along the rows axis (\eg\
	    arcsec)\\ \hline
  \end{tabular}
\end{center}

Of particular importance to VISTA are the \comm{CRVAL1} and \comm{CRVAL2}
cards, which are used to to determine the array indices for the image.  If
these are changed without care, it could cause VISTA to crash.  If these cards
are changed for any reason, you need to copy the image buffer into itself
(\eg\ if you change them in the image header of Buffer 1, you need to issue
the command: \comm{COPY 1 1}).  This operation will reset the internal array
index pointers and keep VISTA happy.  VISTA is internally consistent in its
use of these cards, but there is demonstrated collision with some data from
other institutions.  For example, 2-D images which are wavelength calibrated
along one axis brought to Lick from ESO caused a fair amount of havoc. An
exception is within the plotting commands (\comm{PLOT} and \comm{CONTOUR})
which make use of the FITS coordinate cards as defined by the standard.
Examples of this is given in the chapter on ``dirty tricks''.

The principal point here is to keep in mind that VISTA works internally with
array indices (\ie\ pixel number), not physical coordinates.  Thus, if you add
together two spectra that are wavelength calibrated together, they will be
added {\it pixel-by-pixel}, not by wavelength.  Unless the spectra overlap in
pixels and wavelength simultaneously, the addition of the two will create
garbage. As can be seen from the above table, VISTA does not use physical
coordinate FITS cards in a manner consistent with the definitions outlined in
the 1981 standard paper.  This constitutes VISTA's primary weakness.

%\end{document}
%\end
