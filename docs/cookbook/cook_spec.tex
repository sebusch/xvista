%
%  VISTA Cookbook;  Spectral Image Reduction Chapter (6 for now)
%
%  Last Revision:  1988 August 30
%
%  Note:  Throughout this chapter, subsections are all labelled for
%         cross-referencing.  The base reference key is:
%
%                        sec:sp___
%


%\documentstyle {manual}
%
%\input sys$user:[rick.thesis]thesismacros.pogge
%
%\newenvironment{command}{\begin{center}
%\begin{list}{\tt GO:}{\setlength{\rightmargin}%
%{\leftmargin}}\tt\singlespace }{\end{list}\end{center}}
%
%\def \comm#1{{\tt #1\/}}
%
%\begin{document}
%\setcounter{chapter}{5}

\chapter{Spectral Data Reduction}

\section{Basic Techniques}

This section describes the basic techniques used to reduce long-slit
spectroscopy like that obtained by the Cassegrain spectrographs and the
non-echelle coud\'e spectrographs.  Only the most basic procedures are
considered here, with the following section serving to cover more advanced
reduction techniques and details which will not generally affect the average
user.  It is recommended, however, that even casual users at some point read
the advanced section so that they can be sure that they are not missing some
technique which is important for them.

The main command of interest in spectral reduction is the \comm{MASH} command,
which takes a two-dimensional long-slit image and creates a one-dimensional
spectrum from a section of it.  Before this is done, of course, the usual
treatment of baselining and flat-fielding the two-dimensional image must be
applied (described in the beginning of this manual).  The form of the
\comm{MASH} command is:
\begin{command}
 \item MASH dest src SP=s1,s2 BK=b1,b2 COLS SKY=s REFLAT
\end{command}
\noindent
There are some other keywords which might prove useful --- see the manual for
details.  The \comm{SP=s1,s2} keyword specifies the columns (or rows) which
define the object spectrum.  More than one \comm{SP=} keyword can be used to
co-add several regions of spectrum, but in general only one region is
specified at a time.  The \comm{BK=b1,b2} keyword specifies the background to
be scaled and removed from the spectrum.  Usually this contains sky which must
be removed but even in coud\'e spectra, where there may not be a large sky
contribution, there is often a constant background across most of the CCD
which should be removed.  More than one background region can and often is
specified, usually on either side of the spectrum region.  The \comm{MASH}
command is somewhat clever about the \comm{SP=} and \comm{BK=} windows. If a
given column (or row) is specified in {\it both} \comm{SP=} and \comm{BK=}
keywords then the spectrum designation is given priority and it is used only
as a spectrum column (or row).  This allows the user to use a form like :
\begin{command}
 \item MASH 10 1 SP=314,324 BK=284,354 REFLAT SKY=11 COLS
\end{command}
\noindent
where the background specified in the single \comm{BK=} keyword will actually
define two background regions, columns $284-313$ and $325-354$, straddling the
spectrum.

Before we describe the other keywords a word is in order about how best to
locate the spectrum and background regions.  A simple \comm{PLOT} command
using the \comm{RS=} keyword (or the \comm{CS=} keyword if the spectrum runs
along rows) will usually do the trick. From this the columns at which the
spectrum become visible above the background can be determined, and the
regions where the background is relatively flat can similarly be found.  Note
that depending on the application different criteria for the spectrum rows
might be used.  Some users prefer to define the spectrum ``window'' when the
crosscut through the object's spectrum has risen to 10\% of its peak value,
but the individual user will have to determine what is best for his or her
needs.

Now back to the other \comm{MASH} keywords.  The \comm{COLS} keyword tells
\comm{MASH} that the spectrum lies along columns; otherwise \comm{MASH}
assumes that the spectrum lies along rows on the CCD.  The \comm{REFLAT}
keyword is optional.  It tells \comm{MASH} to use a parabolic fit to each
background row (or column) instead of simply taking an average.  This usually
improves the subtraction of bright, narrow night sky lines, but occasionally
\comm{REFLAT} will actually make things worse.  This can happen, for example,
if two narrow but widely separated background windows are being used.  In such
a case it is clear that a parabola is not well-constrained in the region
between the backgrounds, where presumably it is most critical to the spectrum
subtraction.  Also note that \comm{REFLAT} modifies the original image by
dividing each row (or column) by the parabolic fit to that row's background
pixels.  The \comm{SKY=} keyword is used to place the {\it background}
spectrum into a buffer.  This can then be used by the \comm{SKYLINE} command
to remove first order wavelength shifts induced by spectrograph flexure.

\subsection{Wavelength Calibration}

Another important step in the reduction is the wavelength calibration of the
spectrum.  Very high-precision calibration may rely on rather ``exotic''
techniques such as absorption cells in the light path, Edsel-Butler
interference bands, etc.  These will not be discussed here, nor in the
advanced techniques section, as they are the realms of radial velocity experts
and others.  Ordinarily, the user will have one or more exposures of
wavelength calibration lamps, usually containing some gas or gases like
mercury, helium, argon, neon, iron, thorium, and so on. These images must be
reduced to spectra using the \comm{MASH} command, although in general there is
no need to flat-field the image, and no background specification is used in
the \comm{MASH} command.  An example of a wavelength calibration spectrum is
shown in
Figure~\ref{fig:wavecal}.

\begin{figure}[t]
   \centering
   \vspace{4.0in}
   \caption[]{\label{fig:wavecal} An example of a wavelength calibration
	      image.}
\end{figure}

With one or more of these calibration spectra available we are ready to try to
identify the lines.  It is assumed that an appropriate wavelength list has
already been written into a disk file (see the manual for how to do this if a
list does not already exist for you), and that you have a rough idea as to the
dispersion and central wavelength of the spectra.  Then use the \comm{LINEID}
command:
\begin{command}
   \item LINEID 10 FILE=wavelist CEN=5760 DISP=4 TTY INT
\end{command}
\noindent
where ``wavelist'' is the wavelength line list file, 5760 {\AA} is the
approximate central wavelength in this hypothetical case, and 4 {\AA} is the
dispersion.  The routine will try to identify the lines in the spectrum using
the specified central wavelength ($\pm 100$ pixels) and the specified
dispersion ($\pm 5$\%).  The \comm{TTY} keyword prints out the final line
identifications, while the optional \comm{INT} keyword allows the user to edit
the line list, removing lines that the program misidentifies and adding lines
that it misses.  The line identification is usually quite automatic and
doesn't require the interactive line list editing, but difficult cases can
occur and can require substantial user intervention.  Note that if two or more
sets of different wavelength calibration spectra are to be used there are two
ways to combine the fits.  The first is simply to add the spectra together and
do a \comm{LINEID} on the summed calibration spectrum.  The other is to use
multiple \comm{LINEID}s and after the first one use the \comm{ADD} keyword to
each new \comm{LINEID} command to append the new line list to the previous
one.

Now that \comm{LINEID} has provided a list of identified lines with their
expected wavelengths and observed pixel values the \comm{WSCALE} command fits
a polynomial to these points to provide the actual wavelength calibration. The
form is:
\begin{command}
   \item WSCALE 10 ORD=3 INT
\end{command}
\noindent
where \comm{ORD=3} specifies a polynomial of order 3 (i.e. a cubic) and
\comm{INT} allows the interactive manipulation of line weights.  The
particular circumstances of a set of wavelength calibration spectra will
determine the order of the fit which the user wants to use, although the
program limits this to less than 8. When in doubt, start at a higher order
(for instance 5) and see if the highest order terms in the fit are
significant, using the uncertainties in each parameter which are calculated
and printed in \comm{WSCALE}. If these terms are not significant then you may
lower the order of the fit and try another \comm{WSCALE}.  On the other hand,
for cases in which there is a paucity of calibration lines in one or more
regions of the spectrum it may be more prudent to use a lower-order fit so
that where the polynomial is poorly constrained the high-order terms do not
send it into large oscillations.  The user will have to make his or her own
decision in the end.

\comm{WSCALE} contains a (rather conservative) automatic line-rejection
feature, but it is usual to use the \comm{INT} keyword to remove lines that
are not fit well.  The \comm{WSCALE} command will print out a list of the
lines, their weights in the fits and the residuals from the polynomial fit.
Then it will ask for a line to be changed (specified by an index number so
that you don't have to type in the whole wavelength) and then the new weight
to give that line. A weight of zero will completely remove the line from the
fit. Entering ``$-1$'' for the line index to change will either return the
routine to make a new fit with the altered line weights, or if no line weights
were changed then it will assume that you are finished and will calculate the
reverse fit (i.e. pixel as a function of wavelength instead of vice versa).
Note that it is actually the {\it reverse} fit which is used by the
\comm{ALIGN} command (discussed below) which re-bins the data onto a linear or
logarithmic wavelength scale.  Once a satisfactory fit has been attained the
wavelength calibration spectrum must be saved, as the coefficients of the fit
are stored in the header of the spectrum that was specified in the
\comm{WSCALE} command.  To copy these coefficients from the calibration
spectrum to another uncalibrated spectrum use the \comm{COPW} command (e.g.
\comm{COPW 12 10}).

\subsection{Linearizing the Wavelength Scale}

Quite often the user wishes to re-bin his or her data onto a linear (or
logarithmic in some cases) wavelength scale.  This is done via the
\comm{ALIGN} command.  There are a number of options in \comm{ALIGN} which
will not be covered here; the user should look over the help manual for
details.  It should be mentioned here that \comm{ALIGN} also converts the data
from counts to counts per second per Angstrom, unless the \comm{NOFLUX} keyword
is specified. Always keep this in mind, because often seeming ``bugs'' which
change the scale of spectra can be traced to this feature of \comm{ALIGN}.
Also note that, as mentioned above, \comm{ALIGN} uses the {\it reverse}
wavelength fit (i.e. pixel as a function of wavelength).

More immediately relevant options include the \comm{LGI} keyword. By default
interpolation is done with a sinc interpolation scheme which more or less
preserves the frequencies in the data.  But many users find that the sinc
technique leads to oscillations often called ``ringing,'' common to many
Fourier-based routines.  To avoid this the \comm{LGI} keyword invokes a
4-point, flux-conserving Lagrangian interpolation.  The other available
keywords in \comm{ALIGN} allow the user to remove (or put in) a velocity or
redshift, a shift in pixel space, and other special features.

If the \comm{SKY=} keyword in \comm{MASH} is being used to find a first-order
correction to the wavelength calibration by using the night-sky lines then the
spectrum buffer containing the sky should be \comm{ALIGN}'ed and the
\comm{SKYLINE} command used:
\begin{command}
   \item ALIGN 12 LGI
   \item SKYLINE 10 12
\end{command}
\noindent
assuming that the sky is contained in buffer 12 and the spectrum to be
adjusted is in buffer 10.  \comm{SKYLINE} looks for a few night-sky lines,
mostly in the yellow part of the spectrum, and compares their observed
wavelengths with their expected wavelengths.  A mean shift from these
night-sky lines is calculated and the starting wavelength of the object's
spectrum is adjusted.

\subsection{Flux Calibration and Extinction Correction}

Once a wavelength scale has been associated with a spectrum (whether or not it
has been \comm{ALIGN}'ed) the \comm{EXTINCT} command can be used to correct
for the atmosphere's extinction.  The extinction curve appropriate for Mount
Hamilton is the default, but other extinction curves (e.g. for CTIO) may be
available.  This command then converts observed flux (or counts) to flux (or
counts) above the Earth's atmosphere.

At the Cassegrain focus users are very often concerned with determining an
absolute flux calibration of their spectra.  This is usually done by observing
a flux standard and \comm{MASH}'ing, \comm{ALIGN}'ing, and correcting its
spectrum for atmospheric correction as usual.  Then the command
\comm{FLUXSTAR} can be used to create what is called a ``flux curve,'' which,
when multiplied by an extracted, extinction-corrected spectrum of an object,
will yield a flux-calibrated spectrum.  The \comm{FLUXSTAR} command is of the
form:
\begin{command}
   \item FLUXSTAR 14 FEIGE56 SYSC
\end{command}
\noindent
where 14 is the buffer containing the flux standard, \comm{FEIGE56} is the
file name containing the monochromatic magnitudes of the flux points (see the
help manual for how to set one of these up if necessary), and \comm{SYSC}
tells the program to create a particular type of flux curve.  There are three
types of flux curves which can be created.  The \comm{SYSB} keyword (also the
default if no option is specified) takes the flux points and calculates the
correction (known flux divided by measured counts) at each point within the
spectrum.  A spline is then fit through the flux points and the resulting
smooth curve is the correction curve (called the ``flux curve'').  Note that
the spectrum buffer which is specified in the \comm{FLUXSTAR} command is
replaced by the flux curve, so you may want to copy the standard star spectrum
to another buffer before using this command.  Now any extracted spectrum can
be multiplied by this curve to obtain the ``fluxed'' spectrum of the object.
Some of the strong atmospheric absorption bands in the near-infrared (the A-
and B-bands) are saturated (optically thick) and hence they are not removed by
sky subtraction like most other night-sky features.  To first order they can
be removed using the \comm{SYSC} option, which clips out those parts of the
spectrum in the standard star containing the A- and B-bands and splices them
into the spline flux curve.  An example of this is shown in
Figure~\ref{fig:fluxcal}.   This technique works to a certain extent but the
strengths of these bands varies with the O$_2$ content along the
line-of-sight, and this can change rapidly from point to point in the sky and
from one moment to the next.

\begin{figure}[t]
   \centering
   \vspace{4.0in}
   \caption[]{\label{fig:fluxcal} Example of Flux Curve.}
\end{figure}

More sophisticated techniques are discussed in the advanced section.  The
\comm{SYSA} option is a little-used option which essentially uses the entire
standard star spectrum, inverts it, and then uses the \comm{SYSB} spline fit
to normalize it. Hence some of the features in the standard star spectrum are
retained (inverted) in the flux curve. The usefulness of this option is very
limited and it will not be discussed further. The choice of flux stars and
techniques for taking the data is discussed further in the next section.

\section{Advanced Techniques}

\subsection{Flat-Field Notes}

The treatment of flat-fields also requires some discussion.  There are two
types of effects which the flat-fielding procedure is expected to remove.  The
first, most important effect is the pixel-to-pixel variation in sensitivity.
The second type of effect is more global, lower frequency variation in
sensitivity which might be due to actual quantum efficiency variations, blaze
functions, interference fringes, vignetting, etc.  It should be stressed that
the pixel-to-pixel variation is usually the most important effect to account
for with a flat-field, because there is usually some other technique such as
using a flux curve which will remove the lower frequency patterns.  In this
sense it is sometimes desirable to actually {\it remove} the low frequency
variations in the flat-field and retain only the pixel-to-pixel effects.  This
technique is called flat-field normalization, and is most often done {\it
along} the dispersion direction.  A \comm{MASH} may be used to obtain the
shape of the flat-field along the dispersion direction, which may then be
smoothed or fit with a spline or polynomial.  This smoothed spectrum is then
\comm{STRETCH}'ed into an image and the original flat-field is divided by this
smoothed version.

The \comm{EXTRACT} command discussed below actually requires a normalized flat
field, because that is the most straightforward way of keeping track of the
actual number of electrons which have fallen into any given pixel. Procedures
which calculate the quantum efficiency of the spectrograph also often use a
normalized flat.  Another use is when filters have been used in the flat-field
(usually to suppress the red end of the spectrum and get a more color-balanced
spectral shape) which have rather high frequency structure in their spectral
shape.  However, {\it care must be exercised}! There are real low-frequency
effects which you probably {\it don't} want to remove, such as interference
fringes in the near-IR and certain efficiency effects in gratings. For
instance, the grating efficiency as a function of wavelength is dependent on
the {\it polarization} of the light striking the grating, and often this
efficiency has frequencies in it which are too high to be properly removed by
the flux curve.  This is also an appropriate place to mention that when trying
to retain the highest accuracy with grating spectrographs, it is probably a
good idea to avoid using some standard stars like Hiltner 102 which are also
highly polarized.  The flux measured will depend to a small extent on the
interaction between the polarized starlight and the grating's intrinsic
polarization, and hence on the spectrograph's orientation.

\subsection{\comm{MASH} alternatives: \comm{SPECTROID} and \comm{EXTRACT}}

While \comm{MASH} is the most commonly used method of changing a 2-D long-slit
image into a one-dimensional spectrum, there are a couple of alternatives.
The most useful of these is the \comm{SPECTROID} command, which performs two
functions.  First it will calculate the centroid of the spectrum, using a
spectrum window and a set of background window specifications.  If the
\comm{NOMASH} keyword is used then the command will return this centroid as a
function of column number as the spectrum.  (Note that \comm{SPECTROID} only
works when the spectrum runs across rows, so the \comm{ROTATE} command may
have to be used to place the image into the proper orientation first.)  This
centroid can be valuable by itself or it can be used, smoothed or as is, as a
``model'' for a fainter spectrum on another image. An example of a centroid is
shown in Figure~\ref{fig:centroid}.  Users have thought up many uses for this
command.

\begin{figure}[t]
   \centering
   \vspace{4.0in}
   \caption[]{\label{fig:centroid} An example of a centroid derived from the
      \comm{SPECTROID} command with the \comm{NOMASH} keyword.}
\end{figure}

If the \comm{NOMASH} keyword is {\it not} used then the command will continue,
and using the centroid as a function of column which the program has already
calculated it will extract the spectrum as specified in the command line.  In
contrast with \comm{MASH}, \comm{SPECTROID} interpolates for fractional
pixels. It is useful for many things, most obviously for extracting spectra
which are tilted or curved across the CCD.  It is the main utility used in the
echelle reduction because of this capability.  It is also useful for mapping
distortions.  (Use a night-sky line or a wavelength calibration line as a
``spectrum'' to get an idea of the curvature of these perpendicular to the
dispersion direction.)  It can calculate the distribution of a particular
emission line along the length of a slit, and it can even be used to calculate
higher order moments of the light distribution as a function of column number,
using the \comm{MOMENTS} keyword.

Another command of perhaps more limited usefulness is the \comm{EXTRACT}
command.  This is an ``optimal extraction'' routine patterned after that by
Keith Horne (see {\it Publ. A.S.P.}, {\bf 98}, 609) and is used for extracting
very low signal-to-noise spectra which are dominated by shot noise from the
sky.  It works best for continuum sources rather than pure emission-line
sources.  \comm{EXTRACT} has a number of useful features, and uses the gain
and readout noise of the CCD to estimate the expected noise for each pixel in
the original image.  Because of this it requires a normalized flat-field
(discussed above) so that the number of counts will be approximately conserved
during the flat-fielding.  \comm{EXTRACT} has the ability to automatically
reject points from the operation, if they are farther away from the
parametrized fit than is expected from the calculated uncertainty in each
pixel.  This feature includes rejection of points from the spectrum itself,
something which is not easy to do with the \comm{ZAP} command in many cases.

The algorithm which \comm{EXTRACT} uses is beyond the scope of this cookbook,
but in brief it parametrizes the point spread function (PSF) using
polynomials.  It also parametrizes the background with polynomials, and should
a point fall more than a certain tolerance away from its respective fit (using
the calculated uncertainty) then it is rejected and the fit is repeated.
Because of the way in which the parametrization is done there are limitations
to the use of \comm{EXTRACT}.  In particular highly tilted spectra cannot be
correctly extracted with this command, although highly creative use of the
\comm{SHIFT} command with a model may allow the user to ``fake it.''  Also, it
is known that the polynomial parametrization of the PSF is not very accurate
with high flux levels, and hence any spectrum which is well-exposed should be
created with \comm{MASH} or \comm{SPECTROID}. The value of \comm{EXTRACT} in
improving the signal-to-noise in a spectrum is only manifest in {\it very}
faint spectra, those less than 5\% of the night sky, and hence it is not used
very often.

\subsection{Spatial Distributions and 2-D Distortion Corrections}

One of the nice features of long-slit spectroscopy is the ability to map out
the intensity of a particular line or other feature as a function of position
along the slit.  To do this simply treat the line as a spectrum and the slit
direction as the dispersion direction.  Then \comm{MASH} or \comm{SPECTROID}
can be used to extract the spatial distribution of the feature.

The \comm{SPECTROID} command can also be used in its \comm{NOMASH} mode to map
out the center of a line along the slit, which may be useful, for example, in
showing the velocity of a feature as a function of position along the slit.
It can also be used to map the distortions of the instrument along both the
dispersion and the slit directions.  But no one but Jesus knows how to do
this.

\subsection{``Re-fluxing'' and ``Atmospheric Correctors''}

The normal observing procedures usually do not allow for a particularly
accurate absolute flux calibration because of slit losses and differential
atmospheric refraction.  The former effect is by far the larger, and does not
adversely affect the {\it relative} flux calibration, and hence line ratios.
Normally observers are not overly concerned with the absolute flux levels, as
long as the relative levels are accurate. However, there are observing
techniques which allow quite accurate absolute fluxing of objects.  In general
the observer must take a spectrum of the object using a ``normal'' slit width
to retain spectral resolution.  But then a shorter, ``wide-slit'' observation
can be taken in which the slit width is increased to a size which will
encompass the total flux from the object.  The division of the smoothed
wide-slit spectrum by the smoothed normal spectrum yields a
wavelength-dependent slit loss correction.  This technique is often called
``re-fluxing.''  In this case the flux standards taken throughout the night
must also be taken with a wide slit to assure that all of {\it their} light
has entered the slit.

As mentioned above there are certain night sky absorption features which are
not easily removed from the spectrum with normal sky subtraction techniques.
In particular, the water vapor content of the atmosphere changes dramatically
from hour to hour along one line of sight through the atmosphere, as well as
from one line of sight to another.  Hence, atmospheric correction of these
water absorption features is often done using a star which has few lines in
the spectral region of interest and which is as nearby in the sky as possible
to the object to be corrected.  The spectrum of this ``atmospheric standard''
must be taken close in time to the object's spectrum to assure accurate
results.

%\end{document}
%\end
