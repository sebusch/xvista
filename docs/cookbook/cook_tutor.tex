%
%  VISTA Cookbook;  Brief VISTA Tutorial (Chapter 2)
%
%  Began:  1988 July 3
%
%  Last Revision:  1988 August 9
%
%  Note:  Throughout this chapter, subsections are all labelled for
%         cross-referencing.  The base reference key is:
%
%                        sec:tut
%

%\documentstyle {manual}
%\input sys$user:[rick.thesis]thesismacros.pogge
%
%\newenvironment{command}{\begin{center}
%\begin{list}{\tt GO:}{\setlength{\rightmargin}%
%{\leftmargin}}\tt\singlespace }{\end{list}\end{center}}
%
%\def \comm#1{{\tt #1\/}}
\def \exfile#1{{\tenit #1\/}}
%\def \hitkey#1 {$\langle${\tt #1}$\rangle$\/}

%\begin{document}
%\setcounter{chapter}{1}

\chapter{A Brief VISTA Tutorial}

This chapter presents a brief tutorial of basic VISTA commands for the new
user.  More experienced users can pass this section over, although it may
serve as a useful reference to basic procedures when you're so far into it
that you start to forget the simple stuff.

The first section of the VISTA Help Manual gives the full details of what is
needed to run VISTA, and some of the basic command syntax.  Rather than
regurgitating the contents of the help manual in altered form, this tutorial
is a simple collection of illustrative tasks making use of the most basic
VISTA commands.  As with all chapters, we will start out simple and progress
towards more sophisiticated commands.  These are commands which are general to
any image processing chore.  Before reading this section, read over the first
part of the VISTA Help Manual.  Make sure you understand all about pixels,
rows, column, image buffers, image headers and the like.  If this stuff is
greek to you.  Stop now and read the first part of the manual.

\section{Starting VISTA}

\subsection{Running VISTA}

We will assume that your local VISTA custodian has followed the suggested
installation and had the system manager define the command word ``xvista'' or
put in an appropriate link so that this command will
run VISTA for you, no fuss, no muss.  If your site is different, then you'll
need to find your VISTA custodian and ask them:

\begin{enumerate}
      \item Where the local VISTA executable lives?
      \item What it is called?
\end{enumerate}

%\noindent
%On a VMS machine, you would type something like:
%
%\begin{verbatim}
%      $ RUN [VISTA.BASE]VISTA
%\end{verbatim}
%
%%\noindent
On a Unix machine, you would type something like:

\begin{verbatim}
      % xvista
\end{verbatim}

\noindent
The exact directory syntax may vary from site to site.

VISTA should respond by erasing the screen (depends a little on the type of
terminal or workstation you are on), and print the welcome message.  After the
welcome message (informing you that this is indeed VISTA you're running and
not MONGO or some other damn fool thing), you will be shown, page by page, the
current VISTA news.  If your custodian is the diligent type, then the news
will tell you of any changes to VISTA that might affect you, warnings about
bugs (there are {\it always} bugs), and other bits of useful information. Make
a habit of reading the news as you begin.  When you're done with a page, hit
the space bar to see the next page.

\subsection{Terminal Type}
\label{sec:tutterm}

Know what kind of terminal you are on.  After the news scrolls past, VISTA
will stop and ask you what kind of graphics terminal you are on.  You should
be on some kind of graphics terminal or fancy workstation.  If not, stop and
go find one.  VISTA really doesn't work too well from a line printer.  Also,
unless you have special drivers or an investment in a good quality terminal
emulator, VISTA won't really work on a PC.  If you never want to do graphics,
you're OK, but VISTA {\it is} an {\it image} processing program, somehow not
looking at your data graphically seems a bit pointless.

VISTA will first tell you what the default devices for your installation are.
Your VISTA custodian should have set these to the most common device when
VISTA was installed on your machine.  If you are on one of the default
devices, simply hit return and VISTA will continue on.

If you are not on the default device, then find out what type of terminal you
{\it are} on.  If you type \comm{?} in response to the \comm{terminal type}
prompt, you will be given a menu of the devices that VISTA supports, along
with a number representing the ``device code.'' If your terminal is not one of
these, then you may be in trouble.  VISTA uses Lick MONGO, and Lick MONGO only
supports those terminal types that we have in hand at UCSC.  (It's awful hard
to support terminals we've never heard of.)  Most (but not all) terminals on
the market these days look and smell like DEC VT100s with Retrographics
boards, so you're probably safe if the exact terminal you are on doesn't show
in the device menu.  Check your owner's manual and proceed with fingers
crossed.

If you don't know, then to avoid a potentially major VISTA hassle, {\it don't
guess}.  Find someone who knows.  If you're on a VT100 and you tell VISTA
you're a Sun 4 WorkStation, VISTA will take your word for it and make you pay
by crashing horribly into your lap.  Don't laugh, half of the ``VISTA doesn't
work'' phone calls seem to be people doing just this sort of thing. VISTA is
pretty good, but it doesn't read minds.  So it goes.

\subsection{Logical Assignments (environment variables)}

Users can customize their VISTA environment by setting  some environment
variables {\it before} running VISTA. THese can be used to specify
default directories to use for storage/retrieval of images, etc. There
is also a very useful environment variable which can be set to refer to
a VISTA procedure file which will automatically be run every time  you
start VISTA (and which can include all of your favorite aliases, 
initializations, etc.

If you are on a Unix machine, let's suppose your account name is \comm{ethel},
and your top-level directory is \comm{/usr/ethel}.  Then you would make the
necessary logical assignments by adding the following lines to either your
\comm{.login} or \comm{.cshrc} file:

\begin{verbatim}
      # VISTA Directory and File Assignments
      setenv V_CCDIR=/usr/ethel/images
      setenv V_SPECDIR=/usr/ethel/spectra
      setenv V_PRODIR=/usr/ethel/procedure
      setenv V_DATADIR=/usr/ethel/vdata
      setenv V_STARTUP=/usr/ethel/procedure/startup.pro
\end{verbatim}

The first four lines define the four necessary storage directories for Images,
Spectra, VISTA Command Procedures, and auxilliary data files respectively.
Remember that Unix is case sensitive, so you must define \comm{V\_CCDIR} and
not \comm{v\_ccdir}, otherwise VISTA will ignore your definition.  Again,
these directories must be pre-existing or VISTA will get upset.  To
create these, you would type in (following the above example):

\begin{verbatim}
      % mkdir /usr/ethel/images
      % mkdir /usr/ethel/spectra
      % mkdir /usr/ethel/procedure
      % mkdir /usr/ethel/vdata
\end{verbatim}

(Where, for this document, the \comm{\%} represents the default C-shell
prompt.  It's what the computer types at you.  On some machines, especially
those in networks, the prompt may be different.)

The fifth line points to a pre-existing VISTA procedure file that contains
VISTA commands to be run at startup time.  It is a simple text file (see
Appendix B for some simple examples of procedure files).  While considered
discretionary by VISTA, they are generally so useful, that you should get in
the habit of making one and using it.  A startup file should contain your
favorite command abbreviations (see Appendix B), special variable assignments,
or simply just erase the screen for you at the start.  The latter is simple
enough, define your file \exfile{[FRED.PROCEDURE]STARTUP.PRO} to have just two
lines:

\begin{verbatim}
      CLEAR
      END
\end{verbatim}

All procedure files must end with the \comm{END} command.  The first character
of a line starts on the first position.


\section{Stopping VISTA}

First learn how to start VISTA.  Second learn how to stop VISTA.  The rest is
rock-n-roll.  There are two ways to stop VISTA.  One neat and the other sloppy
but occasionally useful.

\noindent
To stop VISTA at any time you have the VISTA \comm{GO:} prompt, type the
command:

\begin{command}
      \item QUIT
\end{command}

\noindent
VISTA will stop cold, dispose of all image buffers and variables, and return
you to your computer's operating system.  Simple.

If VISTA should hang up (or you're just mad at it and want to kill it in a
satisfying way), then type a \hitkey{CTRL} \hitkey{\}.  VISTA will terminate
immediately.  Admittedly drastic and a little sloppy, it is good in a pinch
when you've sent VISTA into a tight spin and want to bale out.

\section{The VISTA Command Prompt}

After all the preliminaries are done with, VISTA is ready to accept command
input.  The screen should show the VISTA command prompt:

\begin{verbatim}
      GO:
\end{verbatim}

VISTA commands may be typed in either upper or lower case.  VISTA is not case
sensitive (although in Unix versions, file names are most definitely
case sensitive).  Each command is terminated by hitting the \hitkey{RETURN}
key.  Cases where the command is longer than 75 characters is dealt with
later.  The following subsections cover sets of simple commands.

Some people like to have VISTA ``beep'' at them when it wants command input.
You can make VISTA do this by typing the command:

\begin{command}
      \item BELL Y
\end{command}

\noindent
It has its uses, especially to let you know when a particularly long
processing routine has finished.  Mostly it's a matter of taste.

\section{A Simple VISTA Session}
\label{sec:tutsimple}

This section describes a simple VISTA session in which you will read images
off of tape and perform simple tasks like displaying the images, writing them
to disk files, move them around among VISTA buffers, and so forth.  For this
brief example, we will have the following items in hand:

\begin{enumerate}
      \item Assume VISTA installed on a UNIX computer (like a MicroVAX).
      \item VISTA running on a workstation with X11 display
      \item A bunch of FITS images on disk
\end{enumerate}

\medskip

\noindent{Task 1: \underbar{What Images are on Tape?}}
\nobreak
VISTA reads tapes which have been written in the FITS format (FITS stands for
Flexible Image Transport System).  FITS is a standard format for storing
astronomical data, originally developed for radio data but adopted for 2-D
imaging data.  Presumably it is portable anywhere.  Lick data, and FITS tapes
written with VISTA have proven to be portable to MIDAS, IRAF, and AIPS at many
sites, so this shouldn't be a problem.  If your tape is not a FITS format
tape, then you are out of luck (for example, some Palomar data tapes, Lick IDS
tapes, etc. are not FITS format).

To read a tape, first load it onto the tape drive and, when properly spooled,
put the tape drive on-line.  Check with your VISTA custodian for the unit
numbers that VISTA uses for each of your tape drives (if you have only 1 tape
drive, it should be named \comm{UNIT=0}, but check anyway).

\noindent
To mount a tape on the drive named \comm{UNIT=0}, issue the command:

\begin{command}
      \item MOUNT UNIT=0
\end{command}

\noindent
And VISTA will either verify the mount, or come back with an error message
telling you why it failed.

To find out what images are on a tape, you use the \comm{TDIR} command (for
Tape DIRectory). \comm{TDIR} has two modes, wordy and terse.  If you want all
the gory details, then simply type:

\begin{command}
      \item TDIR UNIT=0
\end{command}

\noindent
And the headers of all of the FITS images will be translated and the essential
details printed onto the screen in a dense format (known as ``full'' format).

We aren't interested in all that, and since we want a printout we can read at
our leisure, we'll do a brief tape directory and redirect the output to an
external file which can be printed.  This is done with the following sequence
of commands:

\begin{command}
      \item TDIR BRIEF UNIT=0 'MYSTERY TAPE' >TAPE.LIST
      \item \$~PRINT TAPE.LIST
\end{command}

The first line is a \comm{BRIEF} tape directory of the FITS tape mounted on
\comm{UNIT=0}, labels the directory as \comm{'MYSTERY TAPE'} and redirects the
output to a file called \exfile{TAPE.LIST}.  The \comm{>} is the key used to
redirect output.  Without the redirection ``arrow,'' the tape directory would
be printed on the terminal screen. The second line issues a DCL (\comm{\$})
command to print the file \exfile{TAPE.LIST} on the MicroVAX line-printer.
Note that everything after the \comm{\$} is not a VISTA command, but a VMS
operating system command.  The \comm{\$} allows you to issue operating system
commands without stopping VISTA execution.

\medskip

\noindent{Task 2: \underbar{Read in 2 Images for Inspection}}
\nobreak
Now that we have the tape directory in hand (there were 41 images), we want to
read in images 10 and 11 and look at them.  Since images 10 and 11 are near
the front of the tape, it is best to rewind the tape using the following
sequence of commands:

\begin{command}
      \item DISMOUNT UNIT=0
      \item MOUNT UNIT=0
\end{command}

A \comm{DISMOUNT} followed immediately by a \comm{MOUNT} is a sloppy but
effective rewind.  VISTA knows how to read backwards to an image it has passed
on a tape, but that can be very slow when it has a long way to go, as in this
case.

Read Image 10 into VISTA buffer 1, and Image 11 into VISTA buffer 2 by
typing the commands:

\begin{command}
      \item RT 10 1 UNIT=0
      \item RT 11 2 UNIT=0
\end{command}

The tape will move forward to the image, and read it off tape into the
requested VISTA buffer.  It translates the image header and prints a verbose
summary of essential information on the screen.

To get a brief summary of the contents of the VISTA buffers, issue the
command:

\begin{command}
      \item BUF
\end{command}

\noindent
VISTA will print a the image dimensions (in rows and columns), the origin of
each image (in rows and columns), the name of the image, and other useful
information.  A more complete listing of the buffers would be gotten by typing
the command:

\begin{command}
      \item BUF FULL
\end{command}

Note that \comm{BUF} is an abbreviation of the full command name,
\comm{BUFFER}.  VISTA recognizes unambiguous abbreviations for all commands
(but {\it NOT} for the keywords making up a command).  If you give it an
ambiguous abbreviation, it will tell you so.  Try it, type:

\begin{command}
      \item B
\end{command}

\noindent
And VISTA will tell you all of the commands that begin with \comm{B} and ask
you to be more specific.

\medskip

\noindent{Task 3: \underbar{Display the Images on the Color Display}}
\nobreak
Let's have a look at the images.  Image display is done using some kind of
color display (typically an AED or some fancy color workstation like a Sun
4/110).  To display the image in buffer 1, issue the command:

\begin{command}
      \item TV 1
\end{command}

VISTA will automatically translate the image intensities into colors and
display the image on the color monitor.  Since we have used no command
arguments for \comm{TV}, we will get the default ``color'' map (Black and
White monochrome), and 4 times the image mean intensity (computed
automatically by the \comm{RT} command) will be used to set the scale of the
mapping (\ie\ 0 will be full black, and 4 times the mean intensity will be
full white).

If black and white doesn't do it for you, you can load a color map.  The
fast an easy way is to use the command:

\begin{command}
      \item COLOR CF=RAIN
\end{command}

\noindent Which will load the ``RAINbox'' color map.  (\comm{CF} stands for
``color file,'' and artifact of Version 1.0). There are other color maps
available, see the manual.  If you issue \comm{COLOR} without arguments, then
it will begin asking for data that it wants to construct a custom color map.
Color maps are arcane, and beyond the scope of this cookbook to go into.

Another way, to avoid typing two commands, is to specify the color map with
the \comm{TV} command.  Don't ever do this to change the color map of the
image currently on the display, that would be a horrible waste of CPU time.
To specify the color map at display time, you would issue the command:

\begin{command}
      \item TV 1 CF=RAIN
\end{command}

\medskip

\noindent{Task 4: \underbar{Plot a Row or Column of an Image}}
\nobreak
Another way to look at your data is to plot along a given row or column.
To plot the image intensities along ROW 250 of the image in buffer 2, you
would issue the command:

\begin{command}
    \item PLOT 2 R=250
\end{command}

\noindent
Default X and Y axis limits will be computed, and the plot will appear on your
terminal screen.  If instead, you see no plot but a stream of apparent
gibberish of characters, you haven't specified the right terminal type (see
\S\ref{sec:tutterm}\ above).

You can change the plotting limits using keywords.  For example, you're only
interested in the data between Columns 200 and 350 along Row 250.  To restrict
the plot to this range, you'd type:

\begin{command}
    \item PLOT 2 R=250 XS=200 XE=350
\end{command}

Now, you only want to see the intensity values between 0 and 255.3, thus you
would type:

\begin{command}
    \item PLOT 2 R=250 XS=200 XE=350 MIN=0. MAX=255.3
\end{command}

Finally, you'd like a hardcopy to post on your door.  To do this you would add
an additional keyword, \comm{HARD} as follows:

\begin{command}
  \item PLOT 2 R=250 XS=200 XE=350 MIN=0. MAX=255.3 HARD
\end{command}

\noindent No plot will appear on your screen, and the hardcopy will
(hopefully) appear on the plotting device.  If it does not, then you may have
to consult with your local VISTA custodian to see if something is not set
correctly.

\medskip

\noindent{Task 5: \underbar{Save the Images in Disk Files}}
\nobreak
Having inspected the images, you want to save them both in Disk files for easy
access later.  Disk files are good for temporary storage during reduction and
analysis, but they eat up a large amount of space.  If you are working on an
account with a fixed disk space quota, that quota may have to be increased if
you want to store many images temporarily on disk.  As with all shared
computer resources, you should be frugal.

Let us store the image in buffer 1 in a disk file named \exfile{OLLIE.CCD},
and the image in buffer 2 as \exfile{STANLEY.CCD}.  To do this, you would
issue the commands:

\begin{command}
      \item WD 1 OLLIE
      \item WD 2 STANLEY
\end{command}

\noindent The \comm{.CCD} extension is automatically appended to the file
name, and the files are written into the image directory you specified at
runtime with the \comm{V\_CCDIR} logical name.  If you wanted to write the
file somewhere else, you would add the directory information onto the front of
the file.

To conserve space as much as possible, the image values are converted into
integers between $-$32767 and +32767, and the scaling coefficients stored in
the image headers.  When the images are read back in from disk, VISTA will
translate the image intensities back into their original values.  If you want
the images to be written in their full REAL*4 precision, you would add the
\comm{FULL} keyword to the \comm{WD} command.

\medskip

\noindent{Task 7: \underbar{Moving Images around Buffers}}
\nobreak
To move the image in buffer 1 to buffer 3, you would type the command:

\begin{command}
      \item COPY 3 1
\end{command}

\noindent In VISTA operations between two buffers, the command syntax is
always of the general form:
\begin{command}
      \item OPERATION <to> <from>
\end{command}

\noindent or equivalently:
\begin{command}
      \item OPERATION <destination> <source>
\end{command}

This resembles so-called Reverse Polish Notation (RPN) popular among owners
of HP hand calculators.  For the more algebraically minded, another way to
think about the order of buffers is:
\begin{command}
      \item OPERATION <buffer to change> <buffer left unchanged>
\end{command}

OK, back to the task at hand (moving images around, remember?).  If you
now issue a \comm{BUF} command, you will have 3 images connected, of
which the images in buffers 1 and 3 are now identical.  You can
get rid of the old image in buffer 1 by issuing the command:

\begin{command}
      \item DISPOSE 1
\end{command}

This will erase the image from memory, and free that space for another image.

Using \comm{COPY} to copy an image into a different buffer which is already
occupied will have the effect of deleting the existing image, and replacing it
with the source image indicated by the \comm{COPY} command.


\medskip

\noindent{Task 8: \underbar{Dismount the Tape and Quit VISTA}}
\nobreak
The first step in finishing a session is to dismount any tapes.  To do this,
issue the command:

\begin{command}
      \item DISMOUNT UNIT=0
\end{command}

\noindent Which will rewind and dismount it.  You will need to physically take
the tape off-line and remove the tape from the drive. The second step, while
not absolutely necessary is illustrative, dispose of the VISTA buffers.  You
would do this by issuing the command:

\begin{command}
      \item DISPOSE ALL
\end{command}

\noindent Finally, stop VISTA by typing the command:

\begin{command}
      \item QUIT
\end{command}

\noindent VISTA will terminate, and return you to the operating system
prompt after a brief pause.

\section{Final Remarks}

VISTA is a system best learned by using.  Typical learning times are a few
days to many months, depending on the individual, and the degree of
sophisitication of the image procession task at hand.  Most new users find it
very useful to ``noodle'' around with VISTA, using a dummy image stored on
disk, and trying various commands out.  VISTA is surprisingly easy to use, and
quite flexible once you're accustomed to it.  It is very easy for a first-time
user to sit down and get an image read in from a raw data tape and displayed
in color right away.

% ----------------------------------------------

%\end{document}
%\end
