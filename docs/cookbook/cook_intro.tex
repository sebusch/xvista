%
%  VISTA Cookbook;  Introduction (Chapter 1)
%
%  Began:  1988 June 7
%
%  Last Revision:  1988 August 30
%
%  Note:  Throughout this chapter, subsections are all labelled for
%         cross-referencing.  The base reference key is:
%
%                        sec:int
%

%\documentstyle {manual}
%\input sys$user:[rick.thesis]thesismacros.pogge

%\newenvironment{command}{\begin{center}
%\begin{list}{\tt GO:}{\setlength{\rightmargin}%
%{\leftmargin}}\tt\singlespace }{\end{list}\end{center}}

%\def \comm#1{{\tt #1\/}}
%\def \hitkey#1 {$\langle${\tt #1}$\rangle$\/}

%\begin{document}

\pagestyle{manual}
\pagenumbering{arabic}
\chapter{Introduction}

\section{Overview of VISTA}

VISTA is an interactive image reduction and analysis package developed at Lick
Observatory.  VISTA development was begun by Richard Stover and Tod Lauer in
the early 1980's to provide software for the reduction of data obtained with
the new CCD cameras coming into use at Mount Hamilton.  General agreement among
the astronomy departments at the other UC campuses led to the adoption of
VISTA as the standard image processing package, and all campuses (at first)
installed similar computer hardware on which is was to be run (Digital
Equipment Corporation VAX/VMS systems AED 512 color image displays).  As a
result, the VISTA package from the start was tailored to this equipment.  A
general overview of the architecture of the VISTA package may be found in an
article by Richard Stover in {\it Instrumentation for Ground-Based Optical
Astronomy: Present and Future}, the proceedings of the 1988 Summer Workshop
held in Santa Cruz. 

Since that time, the VISTA package has taken on a life of its own, spreading
to many parts of the world besides the UC system.  The principal agents of
this spread have been former graduate students, postdocs, and visitors who
have taken VISTA with them when they departed. It has been ported to a UNIX
platform and has been successfully used on Sun workstations (SunOS and
Solaris), DecStations, and DEC Alpha machines.
Non-image graphics (\eg\ line drawings, spectra, etc) is done using Lick
MONGO, a fully portable incarnation of the old Tonry MONGO which has evolved
at Lick Observatory over the years.  Lick MONGO supports a variety of graphics
display terminals, as well as the most common hardcopy devices, including
PostScript laser printers. 

VISTA has no ``philosophy'' {\it per se}.  It is designed to be a small
package (small, that is, by image processing standards) that lends itself
readily to user modification.  As such, it is relatively easy for an
astronomer with a modest command of Fortran to write custom commands to take
care of special image processing needs.  It is not intended to be all
encompassing, like the larger MIDAS or IRAF packages. VISTA does have its
faults, which any long-time user will be glad to tell you all about, and which
you will discover for yourself in time.  Recall that it was produced by
graduate students and research astronomers, not professional programmers, so
it has a nice, homey, slapped together in the garage sort of feel at times (it
can also be incomprehensible as hell, too.  So it goes). 

The general development of routines for the VISTA package has been driven by
the needs of individual researchers, principally graduate students working on
their Ph.D. dissertations (\ie\ driven by that oldest of instincts, self
preservation).  In addition, it is often these graduate students who are
called upon for consultations when problems arise, or to assist visiting
scientists with getting started using the VISTA package to reduce their data.
Since graduate students are not eternal (at least, we're not {\it supposed} to
be), it was decided among the present hard core of VISTA gurus to distill our
experience into a ``VISTA Cookbook'' before we kick off for the boondocks. 
This cookbook is to serve as a guide to basic image reduction procedures with
VISTA for new users and visitors.  The idea was for us not to simply vanish to
a postdoc somewhere and leave the people who have come to depend on us at Lick
``turning slowly in the wind.'' 

%-------------------------------------------------------------------------------

\section{The Cookbook}

This cookbook is designed as a resource for VISTA users interested in
accomplishing basic image reduction chores.  It is divided into 8 chapters,
each of which covers a single topic.  Within each chapter, the discussion
starts at the most basic level, progressing towards more advanced techniques.
It is not all encompassing, but that is impossible, as each person's data will
ultimately require different treatment depending on how it was taken and what
information they wish to derive from it.  The intent is to provide new users
with a set of basic VISTA tools which can be used to develop individually
tailored  image reduction and analysis procedures. 

The format is laid out so as to guide the user as much as possible. Different
typefonts are used to distinguish the running commentary from input required
by the computer and the names of external data files.  They are: 

\begin{center}
  \begin{tabular}{|c|p{4.0in}|}
      \hline
      \multicolumn{2}{|c|}{Cookbook Fonts}\\ \hline
      \multicolumn{1}{|c|}{Font}&\multicolumn{1}{c|}{Use}\\ \hline
      {\tt EXTSPEC}&Computer command input (either VISTA or 
operating system)\\ \hline
      {\tenit HAMPREP.PRO}&Names of external data files\\ \hline
      \hitkey{RETURN} & Particular keys on a terminal keyboard \\ \hline
  \end{tabular}
\end{center}

User input prompts are used throughout the cookbook.  These are meant to
reproduce, as well as possible, what you should see on the screen, and to
remind you that the input is either a VISTA command or an operating system
command (DCL or Unix).  The prompts are: 

\begin{center}
  \begin{tabular}{|c|p{4.0in}|}
    \hline
    \multicolumn{2}{|c|}{Command Prompts}\\ \hline
    \multicolumn{1}{|c|}{Prompt}&\multicolumn{1}{c|}{Meaning}\\ \hline
    {\tt GO:}& The default VISTA command prompt.\\ \hline
    {\tt \$}& The default DCL prompt for the operating system.\\ \hline
  \end{tabular}
\end{center}

It is expected that the cookbook will be used in conjunction with the VISTA
Help Manual.  The help manual is available in the form of a computer printout,
as well as accessible on-line while running VISTA. 

A brief synopsis of the cookbook is as follows:  We begin with a brief
tutorial of basic VISTA commands (Chapter 2) to give new users a place to get
started. With the preliminaries out of the way, we begin our discussion of how
to use VISTA for a variety of image processing tasks.  Chapter 3 is an
overview of basic reduction procedures common to most image processing tasks
(imaging or spectrophotometric data), such as flat fields, dark frames, etc. 
In Chapter 4 we describe the basic VISTA procedures for 2-D image reduction.
Basic spectral data reduction techniques are described in Chapter 5.  An
advanced discussion of reduction techniques for data obtained with the
Hamilton echelle spectrograph at Mount Hamilton is presented in Chapter 6. 
Chapter 7 presents a brief tutorial on the use of command procedures in VISTA
for automating various reduction tasks.  Finally, Chapter 8 is a brief
collection of ``dirty tricks'' that some folks might find useful. 

Following the chapters are a series of useful appendices. Appendix A is a
compendium of all of the VISTA commands organized by function.  A set of
examples of VISTA command procedures, ranging from very simple to rather
complex, are given in Appendix B. Appendix C presents a brief discussion of
FITS header cards and how they are used in VISTA. 

VISTA has been ported to both VMS and Unix (ISI) systems, using a variety of
color image display devices.  Throughout this cookbook, we shall use examples
that are for the most common installation at the time of writing, namely a
UNIX computer using an X11 color graphics display.  
Variations are described in the Help Manuals for the different
flavors of VISTA. 

\section{Final Remarks}

For the future? It is clear that VISTA has a fairly large cadre of
loyal users, and so will probably be supported in some state or another for
the present time.

%-------------------------------------------------------------------------------
%
%\section{The Recipe}
%
%There are no banana recipes in this cookbook.  In fact, there aren't {\bf any}
%recipes in this cookbook.  OK, there's {\bf one} recipe in this cookbook.  It
%couldn't be a cookbook without at {\it least} one, right?
%
%\begin{verbatim}
%
%    Killer Margaritas.
%
%         1 fifth of Tequila (cheap Mexican tequila, the cheaper 
%                            the better, with the worm still kicking)
%         Triple Sec  (Grand Marnier would spoil it)
%         Limes
%         Party Ice
%
%         Coarse Salt
%         1 qt blender.
%         Mason Jars
% 
%   Remove worm (do this before you start drinking).  Cut the 
%   worm in half.  Fill the blender 2/3 full of party ice.  
%   Pour half of the Tequila and as much Triple Sec as you can 
%   stand into the blender.  
%   Put one of the worm halves in the blender.  Squeeze a lime
%   (or two) into the blender.  Don't worry about seeds, they 
%   add texture.  Turn on the blender.  
%   Blend until the ice is crushed.  Turn off the blender.  
%   Pour the Margaritas into Mason Jars rimmed with coarse salt.
%
%   Drink the Margaritas (slowly). 
%
%   Makes 2 batches.
%
%\end{verbatim}
%
%\end{document}
%\end
