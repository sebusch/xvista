\chapter{Getting, Saving, and Printing Non-Image VISTA Data} 

%
% checked keywords & function against current sources
% moved SETDIR and CD to chap6 (needed there before here)
% re-organized the introductory section
% rwp/osu 98Aug24
%

\section{Non-Image Data Files used by VISTA}
\index{Data!Introduction to non-image data files} 
\begin{rawhtml}
<!-- linkto data.html -->
<!-- linkto files.html -->
\end{rawhtml}

In addition to images and spectra, VISTA generates a number of auxiliary
non-image data structures.  Some VISTA commands produce data that are
stored internally in Fortran COMMON blocks.  These data can be saved into
disk files using the SAVE command described below and can be read back into
VISTA at a later time using the GET command.  The PRINT command is used to
show the contents of these auxiliary data structures.

VISTA can also read numeric and alphanumeric data from ASCII text files.
The commands for manipulating these files are:
\begin{example}
  \item[OPEN]{open an ASCII file}
  \item[CLOSE]{close an Open ASCII file}
  \item[READ]{read the next line of an ASCII file}
  \item[SKIP]{designate lines to skip in a file}
  \item[REWIND]{rewind an open ASCII file to the first line}
  \item[STAT]{compute statistics on data in an ASCII file}
\end{example}

%---------------------------------------------------------------------------

\section{GET/SAVE: Get/Save a VISTA Data File}
\index{Data!Getting data file}
\index{Data!Saving data file}
\begin{rawhtml}
<!-- linkto get.html -->
<!-- linkto save.html -->
\end{rawhtml}
\begin{command}
  \item[\textbf{Form: } SAVE data\_keyword=filename {[LOW=lowbad]} 
       {[HIGH=highbad]}\hfill]{}
  \item[GET  data\_keyword=filename\hfill]{}
  \item[data\_keyword]{is a word specifying the type of data
       being saved or get (APER, MASK, PHOT, COO or DAO, or PROFILE),}
  \item[filename]{is the name of the file that is holding
       the data (GET) or will hold the data (SAVE)}
\end{command}

The SAVE and GET commands are used for writing out results produced by
other VISTA routines to disk for future use by VISTA or other programs, and
getting them back again.  Various VISTA commands will generate reduced data
which you may want to save for later analysis.  To SAVE or GET results from
a specific program, specify that program's keyword and a file name.  If no
file name is given, VISTA will ask you for one.  More than one keyword can
be specified at a time.  By default, all files are kept in the data
directory (specified by the V\_DATADIR environment variable) with the file
extensions given below.  These can be overridden if desired.  To examine
the current results that VISTA is storing, use the PRINT command.

At this time VISTA provides for the storage of the following data types:

\begin{command}
  \item[Keyword:\hfill]{}
  \item[APER=name]{Aperture photometry file 'name(.APR)'}
  \item[MASK=name]{Photometry mask file 'name(.MSK)'}
  \item[PHOT=name]{Stellar photometry file 'name(.PHO)'}
  \item[COO= or DAO=]{Reads DAOPHOT style ASCII file (\# x y ) into
       internal photometry file}
  \item[PROF=name]{Surface photometry file 'name(.PRF)'}
  \item[LINEID=name]{Saves the LINEID identifications.}
\end{command}

Note that for DAOPHOT files (COO= or DAO=), you can set the LOWBAD and/or
HIGHBAD data values using the LOW= and HIGH= keywords.

The various formats for the different types of data files are described in
more detail in the commands which make use of them.

Examples:  Assume the data directory is vista/data
\begin{example}
  \item[GET PHOT=ORION\hfill]{loads the photometry file
       vista/data/ORION.PHO to VISTA}
  \item[SAVE MASK=MASK5\hfill]{writes the VISTA mask file
       to vista/data/MASK5.MSK}
\end{example}


\section{PRINT: Print Data Files, Spectra, or Image Subsections}
\index{Data!Printing}
\index{Photometry!Printing results}
\index{Image!Print pixel values in}
\index{Profile!Printing results}
\index{Aperture photometry!printing results}
\index{Spectrum!Print values in}
\index{Spectrum!Print wavelength scale}
\index{Spectrum!Print line identifications}
\index{Image!Print headers of disk images}
\index{Spectrum!Print headers of disk spectra}
\index{Strings!Print defined strings}
\index{Files!Print current file directories}
\begin{rawhtml}
<!-- linkto print.html -->
\end{rawhtml}
\begin{command}
  \item[\textbf{Form: } PRINT {[data keywords]} {[output redirection]}\hfill]
  \item[data keywords]{specify which information is printed}
\end{command}

PRINT will print out a formatted listing of reduced data, spectra or image
subsections.  The output from PRINT appears on your terminal; to send it to
a file use the '$>$filename' construct.  PRINT recognizes the following
keywords:

\begin{example}
  \item[object {[BOX=n]}\hfill]{ Print out the pixel values of object (in
       the subsection specified by box 'n').  If the object is a
       wavelength- calibrated spectrum, the wavelengths will be printed.
  \begin{hanging}
    \item{PRINT 1 BOX=2}
    \item{PRINT \$Q BOX=4 $>$imagesec.dat}
  \end{hanging}
}

  \item[BOXES\hfill]{Print out the sizes, centers, and origins of all boxes
       defined.
  \begin{hanging}
    \item{PRINT BOXES}
  \end{hanging}
}

  \item[PROF {[MEDIAN]} {[MAG]} {[SPIRAL]}\hfill]{ Print out the surface
       photometry profile.  If the MEDIAN keyword is specified, the median
       surface brightnesses, rather than the mean, will be printed. If the
       MAG keyword is specified, the surface brightnesses and total counts
       will be output in magnitude units. If the SPIRAL keyword is given
       (for SPIRAL galaxies), then some extra information, like the disk
       scale length, disk/tot, etc. will be printed - if these values have
       been computed using CORRECT.
  \begin{hanging}
    \item{PRINT PROFILE}
    \item{PRINT PROFILE $>$prof.dat}
  \end{hanging}
}

  \item[PHOT {[BRIEF]}\hfill]{ Print out the stellar photometry results.
       BRIEF produces a short listing.
  \begin{hanging}
    \item{PRINT PHOT }
    \item{PRINT PHOT BRIEF}
    \item{PRINT PHOT $>$photlist.dat}
  \end{hanging}
}

  \item[APER\hfill]{ Print out the aperture photometry results.
  \begin{hanging}
    \item{PRINT APER}
    \item{PRINT APER $>>$aperlist.dat}
  \end{hanging}
}

  \item[DIRECTORIES\hfill]{Print the default directories.
  \begin{hanging}
    \item{PRINT DIRECTORIES}
    \item{PRINT DIRECTORIES $>$dirs.dat}
  \end{hanging}
}

  \item[IMAGES or SPECTRA\hfill]{Print headers for images or spectra in the
       default directory.  (Use PRINT DIRECTORIES to see which directory
       this is)
  \begin{hanging}
    \item{PRINT IMAGES}
    \item{PRINT SPECTRA $>$imlist.dat}
  \end{hanging}
}

  \item[LINEID\hfill]{Print wavelength v. pixel identifications obtained
       with the LINEID command.
  \begin{hanging}
    \item{PRINT LINEID}
    \item{PRINT LINEID $>$lines.dat}
  \end{hanging}
}

  \item[STRINGS\hfill]{Print all defines strings.  See the command STRING
       to define strings.
  \begin{hanging}
    \item{PRINT STRINGS}
    \item{PRINT STRINGS $>$string.txt}
  \end{hanging}
}
\end{example}

\section{OPEN: Open an ASCII Text File}
\index{ASCII files!Opening for reading}
\begin{rawhtml}
<!-- linkto open.html -->
\end{rawhtml}
\begin{command}
  \item[\textbf{Form: } OPEN logical\_name file\_name\hfill]{}
  \item[logical\_name]{Is the name you assign to the file}
  \item[file\_name]{Is the disk file name.  No default VISTA directories or
       extensions are applied to the name before it is opened.}
\end{command}

OPEN attempts to do a Fortran OPEN on the specified file. Such files must
be normal sequential files as might be generated by the editor or various
commands.  If the file is successfully opened the logical\_name is then
assigned to the opened file and all further references to the file are made
using the logical name. If an OPEN is done using a logical name which is
already assigned to an opened file, this old file is closed and the new
file is opened.  When a file is first opened, the first line of the file is
ready to be read.  A detailed example of how to use OPEN and the logical
file name is given in the description of the READ command. 

A maximum of up to five (5) files can be OPEN'ed at once.  See CLOSE to
close a currently open file.

\begin{example}
  \item[OPEN DATA ./mydatafile.dat\hfill]{ Opens the file for reading
       and assigns it the logical name DATA.}
\end{example}

Note that while in Unix all file names are case sensitive, but within
VISTA the logical names assigned to them are case insensitive, thus:
\begin{verbatim}
   OPEN DATA ./myfile.dat
   OPEN data ./myfile.dat
\end{verbatim}
are identical.

\section{CLOSE: Close an Open ASCII Text File}
\begin{rawhtml}
<!-- linkto close.html -->
\end{rawhtml}
\index{ASCII files!Closing}
\begin{command}
  \item[\textbf{Form: }CLOSE logical\_name\hfill]{}
  \item[logical\_name]{is the name of a file previously
       opened for reading by the OPEN command.}
\end{command}

CLOSE allows you to close one of the ASCII text files which you have
previously opened with the OPEN command.  You may occasionally want to do
this because there can be only five (5) OPEN'ed files at one time.  

In general it is a good practice to CLOSE all files OPEN'ed by a procedure
at the end, as files opened by a procedure are not automatically closed
when the procedure terminates.

\section{READ: Read the Next Line of an ASCII Text File}
\index{ASCII files!Reading}
\begin{rawhtml}
<!-- linkto read.html -->
\end{rawhtml}
\begin{command}
  \item[\textbf{Form: }READ logical\_name\hfill]{}
  \item[logical\_name]{is the name of a file previously
       opened for reading by the OPEN command.}
\end{command}

READ will read the next line of the named opened file (the argument of READ
is the logical name of a file previously opened with OPEN).  This line then
becomes the 'current' line for the file and all subsequent references to
the file in arithmetic expressions use the current line.  Each READ causes
a new line to be read in the order in which they appear in the file.
However, it is possible to skip specified lines in the file using the SKIP
command. The following example shows how to use the OPEN, and READ commands
in conjunction with arithmetic expressions.

Suppose you have a file called photometry.dat containing the following four
lines of data:
\begin{verbatim}
     B      V
   100.5  150.3
   110.4  164.9
    75.3  113.6
\end{verbatim}
We could compute B-V using the following simple procedure.
\begin{verbatim}
    OPEN PHOT photometry.dat
    SKIP PHOT 1
    DO I=1,3
       READ PHOT
       BV=2.5*(LOG10{[@PHOT.1]}-LOG10{[@PHOT.2]})
       PRINTF '%I2 %F10.3' I BV
    END_DO
    END
\end{verbatim}
The first line opens the file and gives it the name PHOT.  Since the first
line of the file does not contain numeric data and is just a header for the
columns of data, we use the SKIP command to label line 1 as a line to be
skipped.  We then begin to read the real data. The first time line 4 is
executed it reads line 1 of the file.  It finds that we have marked line 1
as a line to skip so it then reads the next line.

Line 5 then makes two references to the file.  The construction @PHOT.1 is
interpreted in the following way: In the 'current' line of PHOT (the one we
just got with the READ), extract the first word of the line and convert it
into a numeric value.  And, of course, @PHOT.2 refers to the second word on
the current line.  The 'word' indicator can be either a constant as shown
or it can be a variable. So if B=1 and V=2 we could have said @PHOT.B and
@PHOT.V.  Expressions are not allowed: @PHOT.(B+1) is illegal and @PHOT.B+1
means "add 1 to the value of @PHOT.B".

\index{ASCII files!Substituting file line into command line}
\index{Strings!Substituting data file line into command line}

Implicit READing: String substitutions, using the {string\_name}
construction, can also substitute lines from an opened file.  To do so use
the form {logical\_name}, which does an implied READ of the named file and
substitutes the entire line from the file into the command line.  You can
also substitute particular words using the form
{logical\_name.word\_indicator} where the word indicator is an expression
giving the word number.  These substitutions always do a new implicit READ
for each substitution.  The following example shows how to use these
implied READ string substitutions.

In this procedure, the user is asked to give a filename.  The file contains
a list of disk file names for images which are to be processed in some
standard way.  The processed image is written out with the same name.
There is one disk file name per line.

\begin{verbatim}
    STRING FILE '?Enter image name file. >> ' ! Get image name file
    OPEN IMAGES {FILE}  ! Open the name file
    STAT LINES=COUNT{[IMAGES]}  ! Get number of lines in file
    DO I=1,LINES  ! For each image ...
       STRING DISKIM {IMAGES}  ! Get image name from file
       RD 1 {DISKIM}                         ! Read image
       CALL PROCESS                          ! Process it
       WD 1 {DISKIM}                         ! Write out
    END_DO
    END
\end{verbatim}

\section{SKIP: Mark Selected Lines to Skip in a File}
\index{ASCII files!Skipping selected lines}
\begin{rawhtml}
<!-- linkto skip.html -->
\end{rawhtml}
\begin{command}
  \item[\textbf{Form: }SKIP logical\_name line line1,line2 ...\hfill]{}
  \item[logical\_name]{is the name of a file opened with the OPEN command.}
  \item[line]{is an arithmetic expression specifying one line in the file
       to skip.}  
  \item[line1,line2]{are two arithmetic expressions giving a range of lines
       to skip.}
\end{command}

SKIP builds a skip table of lines to be skipped inside the named file.  The
file must have been previously opened using the OPEN command.  Up to 50
skip points are available in the table for each opened file.  Each
individual line skipped counts as one specification and each range of lines
skipped counts as two specifications.  Whenever a file is OPEN'ed, or
reOPEN'ed, its skip table is cleared.  REWINDing a file (see the REWIND
command) does not clear the skip table.  If you just type SKIP
logical\_name without any lines to skip then the table of skipped lines for
the named file is printed.

Lines which are SKIPped in a file can not be read with the READ command or
by string substitution and are not used by the STAT command.  In
particular, note that the number of lines in the file as returned by the
STAT command is the actual number of lines minus any skipped lines.

\noindent{Examples:}
\begin{example}
  \item[SKIP PHOT 1]{Marks line 1 of file PHOT to be skipped.}
  \item[SKIP PHOT 100,120]{Marks the range of lines from 100 to 120 to 
       be skipped.}
  \item[SKIP PHOT 1 100,120]{Does both.}
  \item[SKIP PHOT]{Prints the skip table for PHOT.}
\end{example}
See the example under the READ command for one use of the SKIP command.

\section{REWIND: Rewind an Open File to the First Line}
\index{ASCII files!Rewinding}
\begin{rawhtml}
<!-- linkto rewind.html -->
\end{rawhtml}
\begin{command}
  \item[\textbf{Form: }REWIND logical\_name\hfill]{}
  \item[logical\_name]{is the name of an OPEN'ed file.}
\end{command}

The REWIND command repositions the named file back to the beginning of the
file.  The file must already be opened for reading using the OPEN command.
Lines in the file which are marked for skipping using the SKIP command will
continue to be skipped.

\section{STAT: Compute Statistics of Data in a File}
\index{ASCII files!Statistics and line count}
\begin{rawhtml}
<!-- linkto stat.html -->
\end{rawhtml}
\begin{command}
  \item[\textbf{Form: } STAT variable=function{[expression]}\hfill]{}
  \item[variable]{is a VISTA math variable in which the
       value of the statistic is stored.}
  \item[expression]{is an arithmetic expression which involves
       at least one reference to data in an OPEN'ed ASCII file.}
  \item[function]{is one of the following:}
  \item{MAX :Find the maximum value of the expression.}
  \item{MIN :Find the minimum value of the expression.}
  \item{FIRST :Finds the first value of the expression.}
  \item{LAST :Find the last value of the expression.}
  \item{COUNT :Counts the number of lines in the file.
        In this case 'expression' is a logical file name.}
  \item{LOAD   :  Loads the arithmetic expression from each
       line in the input file into a specified buffer
       using STAT N=LOAD{[buffer,expression]}}
\end{command}

The STAT command can be used to determine information about the data values
in an ASCII file.  For the MAX and MIN functions, the given expression is
evaluated for each line in the file.  For the FIRST function, the
expression is evaluated for the first line in the file and for the LAST
function the expression is evaluated for the last line in the file.  The
COUNT function merely counts the lines in the file. Remember that SKIP'ed
lines (see the SKIP command) are never included in the calculations.  These
STAT functions are not the same as the normal VISTA math functions and can
not be included in other mathematical expressions.

The LOAD function allows the user to load data from an input ASCII file
into a VISTA image buffer. Arithmetic operations may be performed on the
input data before loading into the buffer. Simply specify the desired
buffer and the arithmetic expression to load. The new buffer will
automatically be created.

\noindent{Examples:}
\begin{example}
  \item[STAT LINES=COUNT{[DATAFILE]}\hfill]{ Set the variable LINES to the
       number of lines in the file DATAFILE.  DATAFILE must have been
       opened with the OPEN command.  SKIP'ed lines are not counted.}

  \item[STAT MAXVAL=MAX{[2.5*LOG10{[@PHOT.2]}]}\hfill]{ Evaluates the
       expression 2.5*LOG10{[@PHOT.2]} for each line in the file PHOT and
       sets MAXVAL to have the maximum value.  The file PHOT will be left
       repositioned to the beginning of the file after the STAT command
       completes.}

  \item[STAT N=LOAD{[1,@PHOT.1*@PHOT.2]}\hfill]{ Loads the product of the
       values in the first and second columns of the input file PHOT into
       VISTA buffer number 1.}
\end{example}

